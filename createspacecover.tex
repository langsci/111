%%%%%%%%%%%%%%%%%%%%%%%%%%%%%%%%%%%%%%%%%%%%%%%%%%%%
%%%                                              %%%
%%%     Language Science Press Master File       %%%
%%%         follow the instructions below        %%%
%%%                                              %%%
%%%%%%%%%%%%%%%%%%%%%%%%%%%%%%%%%%%%%%%%%%%%%%%%%%%%
 
% Everything following a % is ignored
% Some lines start with %. Remove the % to include them

\documentclass[output=covercreatespace% long|short|inprep              
%  	        ,draftmode  
		  ]{LSP/langsci}    
  
%%%%%%%%%%%%%%%%%%%%%%%%%%%%%%%%%%%%%%%%%%%%%%%%%%%%
%%%                                              %%%
%%%          additional packages                 %%%
%%%                                              %%%
%%%%%%%%%%%%%%%%%%%%%%%%%%%%%%%%%%%%%%%%%%%%%%%%%%%%

% put all additional commands you need in the 
% following files. If you do not know what this might 
% mean, you can safely ignore this section

\usepackage{url}
\usepackage{graphicx}
\usepackage{xspace}
\usepackage{multirow}
\usepackage{multicol}
\usepackage{lingmacros}
\usepackage{umoline}
\usepackage{setspace}
%\usepackage{tipa}
\usepackage{stmaryrd}
\usepackage{fancybox}
\usepackage{url}

\newcommand{\tdl}[1]{\textit{#1}}
%\newcommand{\myexe}[1]{{\small #1}}
\newcommand{\myexe}[1]{{\normalsize #1}}
\newcommand{\mysub}[1]{$_{\textnormal{\tiny{#1}}}$}
\newcommand{\mysout}[1]{\Midline{#1}\xspace}
\newcommand{\myemp}[1]{\textbf{\underline{#1}}}
\newcommand{\myref}[1]{(\ref{#1})\xspace}
\newcommand{\myS}[1]{\S\ref{#1}\xspace}
\newcommand{\mysec}[1]{\S\ref{#1}\xspace}
%\newcommand{\myurl}[1]{{\small\url{#1}}}
\newcommand{\myurl}[1]{\url{#1}}
\newcommand{\mypage}[1]{(p.\ \pageref{#1})}
\newcommand{\mypp}[1]{(\ref{#1},~p.\ \pageref{#1})~}
\newcommand{\xtab}{\xspace\xspace\xspace}
\newcommand{\lingo}{LinGO\xspace}
\newcommand{\lkb}{\textsc{\small LKB}\xspace}
\newcommand{\pet}{\textsc{\small PET}\xspace}
\newcommand{\ace}{\textsc{\small ACE}\xspace}
\newcommand{\agree}{\textit{agree}\xspace}
\newcommand{\itsdb}{\mbox{\textsf{\lbrack incr tsdb()\rbrack}\xspace}}
\newcommand{\logon}{\textsc{\small LOGON}\xspace}
\newcommand{\vs}{vs.\ }
\newcommand{\nun}{-(\textit{n})\textit{un}\xspace}
\newcommand{\onun}{(\textit{n})\textit{un}\xspace}
\newcommand{\ika}{\textit{i} / \textit{ka}\xspace}
\newcommand{\lul}{(\textit{l})\textit{ul}\xspace}
\newcommand{\wa}{\textit{wa}\xspace}
\newcommand{\ga}{\textit{ga}\xspace}



\title{Modeling\newlineCover information structure in a cross-linguistic perspective}  %look no further, you can change those things right here.
\subtitle{}
\BackTitle{Modeling information structure in a cross-linguistic perspective} % Change if BackTitle != Title
\BackBody{%
This study makes substantial contributions to both the theoretical and computational treatment of information structure, with a specific focus on creating natural language processing applications such as multilingual machine translation systems. The present study first provides cross-linguistic findings in regards to information structure meanings and markings. Building upon such findings, the current model represents information structure within the HPSG/MRS framework using Individual Constraints. The primary goal of the present study is to create a multilingual grammar model of information structure for the LinGO Grammar Matrix system. The present study explores the construction of a grammar library for creating customized grammar incorporating information structure and illustrates how the information structure-based model improves performance of transfer-based machine translation.}
%\dedication{Change dedication in localmetadata.tex}
\typesetter{Sanghoun Song}
%\proofreader{Change proofreaders in localmetadata.tex}
\author{Sanghoun Song}
% \keywords{Information Structure, HPSG, MRS, Individual Constraints, Grammar Matrix}%add 5 keywords
\renewcommand{\lsISBNdigital}{000-0-000000-00-0}
\renewcommand{\lsISBNhardcover}{000-0-000000-00-0}
\renewcommand{\lsISBNsoftcover}{000-0-000000-00-0}
\renewcommand{\lsISBNsoftcoverus}{000-0-000000-00-0}
\renewcommand{\lsSeries}{tdgi} % use lowercase acronym, e.g. sidl, eotms, tgdi
\renewcommand{\lsSeriesNumber}{1} %will be assigned when the book enters the proofreading stage
\renewcommand{\lsURL}{http://langsci-press.org/catalog/book/111} % contact the coordinator for the right number

%<*coverdimen>
\setlength{\csspine}{25.0559784mm} % Please calculate: Total Page Number (excluding cover), usually (Total Page - 3) * 0.0572008 mm
\setlength{\bodspine}{20mm} % Please use BoD's algorithm: http://www.bod.de/hilfe/coverberechnung.html (German only, please contact LangSci staff for help)
%</coverdimen> 


\renewcommand{\lsCoverTitleFont}[1]{\sffamily\addfontfeatures{Scale=MatchUppercase}\fontsize{44pt}{16.00mm}\selectfont #1}

\input{localpackages.tex}
\input{localhyphenation.tex}
%add all your local new commands to this file

\newcommand{\smiley}{:)}
\let\sf\sffamily

% put all additional commands you need in the 
% following files. If you do not know what this might 
% mean, you can safely ignore this section

\newcommand{\tdl}[1]{\textit{#1}}
%\newcommand{\myexe}[1]{{\small #1}}
\newcommand{\myexe}[1]{{\normalsize #1}}
\newcommand{\mysub}[1]{$_{\textnormal{\tiny{#1}}}$}
\newcommand{\mysout}[1]{\Midline{#1}\xspace}
\newcommand{\myemp}[1]{\textbf{\underline{#1}}}
\newcommand{\myref}[1]{(\ref{#1})\xspace}
\newcommand{\myS}[1]{\S\ref{#1}\xspace}
\newcommand{\mysec}[1]{\S\ref{#1}\xspace}
%\newcommand{\myurl}[1]{{\small\url{#1}}}
\newcommand{\myurl}[1]{\textit{#1}}
\newcommand{\mypage}[1]{(p.\ \pageref{#1})}
\newcommand{\mypp}[1]{(\ref{#1},~p.\ \pageref{#1})~}
\newcommand{\xtab}{\xspace\xspace\xspace}
\newcommand{\lingo}{LinGO\xspace}
\newcommand{\lkb}{\textsc{\small LKB}\xspace}
\newcommand{\pet}{\textsc{\small PET}\xspace}
\newcommand{\ace}{\textsc{\small ACE}\xspace}
\newcommand{\agree}{\textit{agree}\xspace}
\newcommand{\itsdb}{\mbox{\sf \lbrack incr tsdb()\rbrack}\xspace}
\newcommand{\logon}{\textsc{\mbox{\small LOGON}}\xspace}
\newcommand{\vs}{\textit{vs}.\ }
\newcommand{\nun}{(\textit{n})\textit{un}\xspace}
\newcommand{\ika}{\textit{i}/\textit{ka}\xspace}
\newcommand{\lul}{(\textit{l})\textit{ul}\xspace}
\newcommand{\wa}{\textit{wa}\xspace}
\newcommand{\ga}{\textit{ga}\xspace}
%\newcommand{\ZHONG}{ZHONG\ensuremath{[\Big|]}}

 
\bibliography{localbibliography,sanghounsong} 

%%%%%%%%%%%%%%%%%%%%%%%%%%%%%%%%%%%%%%%%%%%%%%%%%%%%
%%%                                              %%%
%%%             Frontmatter                      %%%
%%%                                              %%%
%%%%%%%%%%%%%%%%%%%%%%%%%%%%%%%%%%%%%%%%%%%%%%%%%%%%
\begin{document}         
\maketitle                
\frontmatter
% %% uncomment if you have preface and/or acknowledgements

\currentpdfbookmark{Contents}{name} % adds a PDF bookmark
\tableofcontents
%\include{chapters/preface}
\include{chapters/acknowledgments}
\include{chapters/abbreviations} 
\mainmatter         
 

%%%%%%%%%%%%%%%%%%%%%%%%%%%%%%%%%%%%%%%%%%%%%%%%%%%%
%%%                                              %%%
%%%             Chapters                         %%%
%%%                                              %%%
%%%%%%%%%%%%%%%%%%%%%%%%%%%%%%%%%%%%%%%%%%%%%%%%%%%%

\include{chapters/01} 
\include{chapters/02}
\include{chapters/03}
\include{chapters/04}
\include{chapters/05}
\include{chapters/06}
\include{chapters/07}
\include{chapters/08}
\include{chapters/09}
\include{chapters/10}
\include{chapters/11}
\include{chapters/12}
\include{chapters/13}
\chapter{Conclusion}
\label{chapter15}
\setcounter{enums}{0}

\section{Summary}
\label{15:sec:summary}


The present study begins with key motivations laid out
Chapter~\ref{chapter1} for the creation of a computational model of
information structure. Chapter~\ref{2:notes} offers preliminary notes
for understanding the current work.

The first part (Chapter~\ref{chapter3}--\ref{chapter5}) scrutinizes
meanings and markings of information structure from a cross-linguistic
standpoint. Information structure is composed of four components:
\isi{focus}, \isi{topic}, \isi{contrast}, and background. Focus identifies that which is
important and/or new in an utterance, which cannot be removed from the
sentence. Topic can be understood as what the speaker is speaking
about, and does not necessarily appear in a sentence (unlike
focus). Contrast applies to a set of alternatives,\is{alternative set}
which can be realized as either focus or topic. Lastly,
\isi{background} is defined as that which is neither focus nor
topic. There are three means of expressing information structure;
prosody,\is{prosody} lexical markers,\is{lexical markers} and
syntactic positioning.\is{syntactic positioning} Among them, the
current work is largely concerned with the last two means, leaving
room for improvement in modeling the interaction between prosody and
information structure as further work.  There are three lexical types
responsible for marking information structure: affixes, adpositions,
and modifiers (e.g.\ clitics).\is{adposition} Canonical positions of
focus include \isi{clause-initial}, \isi{clause-final},
\isi{preverbal}, and \isi{postverbal}. Building upon these fundamental
notions, Chapter~\ref{chapter5} looks into several cases in which
discrepancies in form-meaning mapping of information structure happen.


The second part (Chapter~\ref{chapter8}--\ref{chapter10-4}) proposes
using \isi{ICONS} (Individual CONStraints)\is{individual constraints}
for representing information structure in MRS
(\citealt{copestake:etal:05}).\is{MRS} This is motivated by three
factors.  First, information structure markings should be
distinguished from information structure meanings in order to solve
the apparent mismatches between them. Second, the representation of
information structure should be
underspecifiable,\is{underspecification} because there are
many sentences whose information structure cannot be conclusively
identified in the context of sentence-level, text-based
processing. Third, information structure should be represented as a
\isi{binary relation} between an individual and a clause. In other
words, information structure roles should be filled out as a
relationship with the clause a constituent belongs to, rather than as
a property of a constituent itself. In order to meet these
requirements, three type hierarchies are suggested; \tdl{mkg},
\tdl{sform},\is{sentential forms} and most importantly
\tdl{info-str}.\is{\textit{info-str}}\is{\textit{mkg}}\is{\textit{sform}}
In addition to them, two types of flag features, such as L/R-PERIPH
and LIGHT,\is{lightness} are used for configuring \isi{focus} and
\isi{topic}.\is{L-PERIPH}\is{R-PERIPH} Using hierarchies and features,
the remaining chapters address multiclausal utterances and specific
forms of expressing information structure and also calculate focus
projection via ICONS.\is{focus projection}


The third part (Chapter~\ref{chapter11}--\ref{chapter12}) creates a
customization system for implementing information structure within the
\lingo \isi{Grammar Matrix} (\citealt{bender:etal:10}) and examines
how information structure improves transfer-based multilingual machine
translation.\is{transfer-based} Building on cross-linguistic and
corpus-based findings, a large part of HPSG/MRS-based\is{HPSG}\is{MRS}
constraints presented thus far is implemented in \isi{TDL}. A
web-based questionnaire is designed in order to allow users to
implement information structure constraints within the
\texttt{choices} file. Common constraints across languages are added
into the Matrxi core (\texttt{matrix.tdl}), and language-specific
constraints which depend on the users' choices are processed by Python
scripts and stored into the customized grammar. Evaluations of this
library using regression tests and \isi{Language CoLLAGE}
\citep{bender:14} show that this library works well with various types
of languages.\is{regression test} Finally, an experiment of
multilingual machine translation bears out that information structure
can be used to reduce the number of infelicitous translations
dramatically.





\section{Contributions}
\label{15:sec:contributions}

The present study holds particular significance for general theoretic
studies of the grammar of information structure. First of all, quite a
few languages are surveyed to capture cross-linguistic generalizations
about information structure meanings and markings, which can serve as
an important milestone for typological research on information
structure. Second, the data collection in the corpus study, despite
its small size, offers distributional findings on information
structure. In particular, the data set presents its own implications
given that it is comprised of a set of parallel annotated sentences,
and that the set of included languages presents a diverse set of
information structure properties.



The present study also makes a contribution to HPSG/MRS-based
studies\is{HPSG}\is{MRS} by enumerating strategies for representing
meanings and markings of information structure within the formalism in
a comprehensive and fine-grained way.  Notably, the present study
establishes a single formalism for representation and applies this
formalism to various types of forms in a straightforward and cohesive
manner. Moreover, the current model addresses how information
structure can be articulated within the HPSG/MRS framework and
implemented within a computational system in the context of
\isi{grammar engineering}.



The present study also shows that information structure can be used to
produce better performance in natural language processing systems. My
firm opinion is that information structure contributes to multilingual
processing; languages differ from each other not merely in the words
and phrases employed but in the structuring of information. It is my
expectation that this study will inspire future studies in
computational linguistics to pay more attention to information
structure.


Last but most importantly, the present model makes a contribution to
the \lingo \isi{Grammar Matrix} library. The actual library makes it easy
for other developers to adopt and build on my analyses of information
structure.  Moreover, the methodology of creating libraries I take in
this study can be used for other libraries in the system.  In order to
construct the model in a fine-grained way, I collected
cross-linguistic findings about information structure markings and
exploited a multilingual parallel text in four languages. These two
methods are essential in further advancements in the \lingo framework.





\section{Future Work}
\label{15:sec:future}

This book closes with a brief look at directions for
improvement in the future. 



First, it is necessary to examine other types of particles responsible
for marking information structure.  Not all \isi{focus} sensitive items are
entirely implemented in \isi{TDL} in the current model even for
\ili{English}.\is{lexical markers} \ili{Japanese} and \ili{Korean} employ a
variety of lexical markers for expressing focus and \isi{topic}, which are
presented in \citet{hasegawa:11} and \citet{lee:04}. A few focus
markers in some languages have a positional restrictions. For example,
as shown in Chapter~\ref{chapter4} (\myS{4:sec:lexical}), the clitic
\textit{tvv} in Cherokee signals focus and the focused constituent
with \textit{tvv} should be followed by other constituents in the
sentence. That is, two means of marking information structure operate
at the same time.  It would be interesting to investigate these kinds
of additional constraints in the future.\is{focus sensitive item}


Second, a few more types of constructions related to information
structure will be studied in future work.  The constructions include
echo questions, \textit{Yes}/\textit{No}-questions \citep{king:95},
coordinated clauses \citep{heycock:07}, double nominative
constructions \citep{kim:sells:07,choi:12}, floating quantifiers
\citep{yoshimoto:etal:06,kim:11b}, pseudo clefts
\citep{kim:07},\is{clefting} and \textit{it}-clefts in other languages
in the \isi{DELPH-IN} grammars (e.g.\ \ili{Japanese}
\citep{hiraiwa:ishihara:02,kizu:05} and \ili{Korean}
\citep{kim:yang:09}).


Third,\is{focus projection} the method for computing \isi{focus} projection
in the present study also needs to be more thoroughly examined. There
are various constraints on how focus can be spread to larger
constituents. These are not addressed in the present study, which
looks at the focus projection of only simple sentences in \ili{English}. The
method the present study employs for handling focus projection could be much
reinforced in further studies.


Fourth, it would be interesting for future work to delve into how
scopal interpretation can be dealt with within the framework that the
present study proposes.  Topic has an influence on scopal
interpretation in that topic has the widest scope in a sentence
\citep{buring:97,portner:yabushita:98,erteschik:07}.  MRS employs
HCONS (Handle CONStraints) in order to resolve scope
ambiguity.\is{MRS} Further work can confirm whether HCONS+ICONS is
able to handle the relationship between \isi{topic} and scope resolution.


Finally, the evaluation of multilingual machine translation will be
extended with a large number of test suites. More grammatical
fragments related to \isi{ICONS} will be incorporated into the
\isi{DELPH-IN} resource grammars, such as \isi{ERG} (English Resource
Grammar, \citealt{flickinger:00}), \isi{Jacy}
(\citealt{siegel:bender:02}),  \isi{KRG}
(\ili{Korean} Resource Grammar, \citealt{kim:etal:11}), ZHONG
\ensuremath{[\big|]} (for the \ili{Chinese} languages, \citealt{fan:15a,fan:15b}),\is{ZHONG}
\isi{INDRA} (for \ili{Indonesian}, \citealt{moeljadi:15}), and so forth.





% % copy the lines above and adapt as necessary

%%%%%%%%%%%%%%%%%%%%%%%%%%%%%%%%%%%%%%%%%%%%%%%%%%%%
%%%                                              %%%
%%%             Backmatter                       %%%
%%%                                              %%%
%%%%%%%%%%%%%%%%%%%%%%%%%%%%%%%%%%%%%%%%%%%%%%%%%%%%

% There is normally no need to change the backmatter section
\input{backmatter.tex}
\end{document} 

% you can create your book by running
% xelatex main.tex
%
% you can also try a simple 
% make
% on the commandline

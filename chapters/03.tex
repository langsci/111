\chapter{Meanings of information structure}
\label{chapter3}
\setcounter{enums}{0}


The present study regards \isi{focus}, \isi{topic}, \isi{contrast}, and \isi{background}
as the main categories of information structure, though there is a no
broad consensus on these categories in many previous studies
\citep{lambrecht:96,gundel:99,fery:krifka:08}. (i) Focus
means what is new and/or important in the sentence. (ii) Topic refers
to what the sentence is about. (iii) Contrast applies to a set of
alternatives,\is{alternative set} which can be realized as either
focus or topic.\is{contrast}  (iv) Background is neither focus nor topic.


The main criterion for classifying components of information structure
in the present study is the linguistic forms.  If a particular
language has a linguistically encoded means of marking an information
structure meaning, the information structure category exists in human
language as a cross-cutting component of information structure.  This
criterion is also applied to the taxonomy of information structure in
each language. If a language has a linguistic means of expressing a
type of information structure meaning, the corresponding component is
assumed to function as an information structure value in the language.


The current analysis of information structure meanings builds on the
following assumptions: (i) Every sentence has at least one \isi{focus}, because new
and/or important information plays an essential part in information
processing in that all sentences are presumably communicative acts
\citep{engdahl:vallduvi:96,gundel:99}.  (ii) Sentences do not
necessarily have a \isi{topic} \citep{buring:99}, which means that there are
topicless sentences in human language.\is{topicless}  (iii) Contrast, contra
\citet{lambrecht:96}, is treated as a component of information
structure given that it can be linguistically
expressed.\footnote{\citet[290--291]{lambrecht:96} says ``Given
  the problems involved in the definition of the notion of
  contrastive, I prefer not to think of this notion as a category of
  grammar. To conclude, \isi{contrastiveness}, unlike \isi{focus}, is not a
  category of grammar but the result of the general cognitive
  processes referred to as conversational implicatures.''} (iv)
Sometimes, there is a linguistic item to which neither focus nor topic
are assigned \citep{buring:99}, which is called \isi{background}
(also known as `tail' in the schema of \citeauthor{vallduvi:vilkuna:98})
hereafter.


Building upon the taxonomy presented above, the present study makes
three fundamental assumptions.  First, \isi{focus} and \isi{topic} cannot overlap
with each other in a single clause
(\citealt{engdahl:vallduvi:96}).\footnote{There are alternative
  conceptions to this generalization, such as \citet{krifka:08}: (i)
  Contrast is not a primitive. (ii) Alternatives are always introduced
  by focus.\is{alternative set} (iii) Contrastive topics contain a
  focus.\is{contrastive topic} (iv) Focus and \isi{topic} are thus not mutually exclusive. The
  distributional and practical reasons for not taking these
  conceptions in the current work are provided in the remainder of
  this book from a special perspective of multilingual machine
  translation.}  That means there is no constituent that plays both
roles at the same time in relation to a specific
predicate.\footnote{Chapter~\ref{chapter10-2} looks into two or more
  different information structure values that a constituent can have
  with respect to different clauses (i.e.\ multiclausal
  constructions).} The information structure meaning of a constituent
within a clause should be either, or neither of them
(i.e.\ \isi{background}). Second, as constituents that can receive
prosodic accents presumably belong to informatively meaningful
categories \citep{lambrecht:96}, contentful words and phrases either
bear their own information structure meanings or assign an information
structure meaning to other constituents.  Finally, just as
informatively meaningful categories exist, there are also lexical
items to be evaluated as informatively meaningless. The informatively
void items themselves cannot be associated with any component of
information structure, though they can function in forming information
structure.




\section{Information status}
\label{3:sec:status}




Before discussing information structure meanings, it is necessary to
go over information status such as givenness (i.e.\ new \vs old). It
is my position that information structure interacts with but is
distinct from information status.


Information status has been widely studied in tandem with information
structure \citep{gundel:03}.  For instance, \citet{halliday:67} claims
that \isi{focus} is not recoverable from the preceding discourse because
what is focused is new.  \citet{cinque:77} argues that the leftmost
NPs and PPs in \isi{dislocation} constructions have a restriction on
information status in some languages.  According to
\citeauthor{cinque:77}, in \ili{Italian}, a required condition for
placing a constituent to the left peripheral position of a sentence is
that the constituent should deliver old information. Thus, NPs and PPs
conveying new information cannot be detached from the rest of a
sentence in Italian.  This assumes that information status is
represented by information structure meanings so that new
information bears \isi{focus}, and \isi{topic} is something given in the context.
However, there are more than a few counterexamples to this
generalization \citep{erteschik:07}: New information can occasionally
convey topic meaning, and likewise focus does not always carry new
information.




Definiteness, upon which the choice of determiners is dependent, has
also been assumed to have an effect on articulation of information
structure: Definite NPs carry old information, and indefinite NPs
carry new information. Thus, it has been thought that indefinite NPs
cannot be the \isi{topic} of a sentence, unless used for referencing
generics \citep{lambrecht:96}.  In particular, \isi{topic-comment}
structures have a tendency to assign a topic relation to only definite
NPs. \citet{kuroda:72}, for instance, claims that \wa-marked NPs in
\ili{Japanese}, widely assumed to deliver topic meaning, can be only
translated into definite NPs or indefinite non-specific NPs in
\ili{English}, while \ga-marked NPs (i.e.\ ordinary nominatives) do not have
such a correspondence restriction in translation.  A similar
phenomenon can be found in Chinese.  Chinese employs three types of
word orders such as SVO (unmarked), SOV, and OSV, but the ordering
choice is influenced by the definiteness of the object: The
\isi{preverbal} object in SOV and OSV constructions seldom allows an
indefinite non-specific expression.



\myexe{\eenumsentence{\toplabel{exe:huang:etal:def}
\item\shortex{5}
{wo & zai & zhao & yi-ben & xiaoshuo.}
{I & at & seek & one-\textsc{cl} & novel}
{`I am looking for a novel.' (SVO)}
\item\shortexnt{5}
{*wo & yi-ben & xiaoshuo & zai & zhao.}
{I & one-\textsc{cl} & novel & at & seek (SOV)}
\item\shortex{5}
{*yi-ben & xiaoshuo, & wo & zai & zhao.}
{one-\textsc{cl} & novel & I & at & seek (OSV) [cmn]}
{\mbox{ } \mbox{ } \mbox{ } \mbox{ } \mbox{ } \mbox{ } 
\mbox{ } \mbox{ } \mbox{ } \mbox{ } \mbox{ } \mbox{ } 
\mbox{ } \mbox{ } \mbox{ } \mbox{ } \mbox{ } \mbox{ } 
\mbox{ } \mbox{ } \mbox{ } \mbox{ } \mbox{ } \mbox{ } 
\mbox{ } \mbox{ } \mbox{ } \mbox{ } \mbox{ } \mbox{ } 
\mbox{ } \mbox{ } \mbox{ } \mbox{ } \mbox{ } 
\citep[200]{huang:etal:09}}}}




\noindent However, there are quite a few counterarguments to the
generalization that \isi{topic} is always associated with definiteness.
\citet{erteschik:07} argues that the correspondence between marking
definiteness and topichood is merely a tendency. This argument is
supported for several languages. First, \citet{yoo:etal:07},
exploiting a large English-Korean bilingual corpus, verify there is no
clear-cut corresponding pattern between (in)definite NPs in \ili{English}
and the NP-marking system (e.g.\ \ika for nominatives \vs \nun for
topics or something else) in \ili{Korean}. Thus, we cannot say that
the correlation between expressing definiteness and topichood is
cross-linguistically true. Second, since some languages
(e.g.\ \ili{Russian}) seldom use definite markers, we cannot equate
definiteness with topichood at the surface level.  Definiteness is
presumed to be a language universal. Every language has (in)definite
phrases in interpretation, even though this is not necessarily overtly
expressed in a languages.  Of particular importance to the current
work are the overt marking systems of definiteness in some
languages. For instance, distinctions between different types of
determiners (e.g.\ \textit{the}/\textit{a}(\textit{n}) in \ili{English}) do
not have a one-to-one correspondence with information structure
components.  So that in English, for example, not all NPs specified
with \textit{the} deliver a topic meaning, and NPs with
\textit{a}(\textit{n}) have \isi{topic} meaning in certain circumstances.

I conclude that information status is neither a necessary nor a
sufficient condition for identifying information structure; the
relationship between the two is simply a tendency and quite
language-specific. For this reason, in the present work, I downplay
the discussion of information status, and instead pay more attention
to information structure.



\section{Focus}
\label{3:sec:focus}


\subsection{Definition}
\label{3:ssec:definition-focus}

Focus, from a pragmatic point of view, refers to what the speaker
wants to draw the hearer's attention to
\citep{erteschik:07,fery:krifka:08}.  \citet{lambrecht:96} regards the
basic notion of \isi{focus} in \myref{def:focus}.

\myexe{\eenumsentence{\toplabel{def:focus}
\item{Pragmatic Presupposition: the set of presuppositions
  lexicogrammatically evoked in an utterance which the speaker assumes
  the hearer already knows or believes or is ready to take for granted
  at the time of speech. \citep[52]{lambrecht:96}}
\item{Pragmatic Assertion: the presupposition expressed by a sentence
  which the hearer is expected to know or believe or take for granted
  as a result of hearing the sentence uttered. \citep[52]{lambrecht:96}}
\item{Focus: the semantic component of a pragmatically structured
  proposition whereby the assertion differs from the
  presupposition. \citep[213]{lambrecht:96}}}} 


\noindent In a nutshell, \isi{focus} encompasses what speakers want to say
importantly and/or newly, and this is influenced by both semantics and
pragmatics.  Building upon \myref{def:focus}, the current work
represents information structure within the MRS\is{MRS} (Minimal
Recursion Semantics, \citealt{copestake:etal:05}) formalism.  In the
following subsection, approaches to the taxonomy of focus are provided
on different levels of classification (syntactic, semantic, and
pragmatic). Among them, I mainly adapt \citeauthor{gundel:99}'s
classification, because it is based on linguistic
markings. Ultimately, \isi{semantic focus}
(also known as non-contrastive focus)\is{non-contrastive focus}
and contrastive focus are distinguishably marked in quite a few
languages, and they exhibit different linguistic behaviors from each
other across languages.




\subsection{Subtypes of focus}
\label{3:ssec:subtypes-focus}


\subsubsection{\citet{lambrecht:96}}
\label{3:sssec:lambrecht}

\citeauthor{lambrecht:96} classifies focus into three subtypes
depending on how \isi{focus} meaning spreads into larger phrases; (a-i)
argument focus, (a-ii) predicate focus, and (a-iii) sentential
focus. The main classification criterion \citeauthor{lambrecht:96}
proposes is sentential forms, which suggests that how a sentence is
informatively articulated largely depends on the scope that the focus
has in a sentence.  For argument focus, the domain is a single
constituent such as a subject, an object, or sometimes an oblique
argument.  Predicate focus has often been recognized as the second
component of `\isi{topic-comment}' constructions. That is, when a phrase
excluding the fronted constituent is in the \isi{topic} domain, the rest of
the sentence is an instance of predicate focus.  Sentential focus's
domain is the entire sentence.


This notion has been developed in quite a few studies. For instance,
\citet{paggio:09} offers a type hierarchy for sentential forms,
looking at how components of information structure are articulated and
ordered in a sentence.\footnote{The type hierarchy that
  \citeauthor{paggio:09} proposes is presented in
  Chapter~\ref{chapter8} with discussion about which implication it
  has on the current work.} In the taxonomy of \citet{paggio:09},
there are two main branches, namely focality and topicality.  As
subtypes of focality, \citeauthor{paggio:09} presents narrow \isi{focus} and
wide focus.\is{narrow focus}\is{wide focus}
Note that argument focus is not the same as narrow
focus. The former means that an argument (i.e.\ NP) of the predicate
is marked as the focus of the clause, while the latter means that a
single word is marked as the focus of the clause. Thus, non-nominal
categories such as verbs, adjectives, and even adverbs can be narrowly
focused. The same goes for the distinction between predicate focus and
wide focus. Predicate focus literally means that the predicate plays
the core role of focus, and the focus is spread onto the larger VPs.
Wide focus and predicate focus both involve \isi{focus projection},
but the core of wide focus can be from various lexical categories,
including nominal ones (e.g.\ common nouns, proper names, pronouns,
etc.). In other words, argument focus is a subset of narrow focus;
predicate focus is a subset of wide focus, and a narrow focus is not
necessarily an argument focus; a wide focus does not necessarily
involves a predicate focus.\is{wide focus}


\subsubsection{\citet{kiss:98}}
\label{3:sssec:kiss}

\citeauthor{kiss:98}, in line with alternative semantics
(\citealt{rooth:92}), suggests a distinction between (b-i)
identificational \isi{focus} and (b-ii) informational focus.

\myexe{\enumsentence{\toplabel{def:kiss} An identificational focus
    represents a subset of the set of contextually or situationally
    given elements for which the predicate phrase can potentially
    hold. \citep[245]{kiss:98}}}

\noindent \myref{def:kiss} implies that identificational focus has a
relation to a powerset of the set consisting of all the elements in
the given context. Thus, the elements in the \isi{alternative set} of
identificational foci are already introduced in the context, while
those of informational foci are not provided in the prior context.
The difference between them can be detected in the following sentences
in Hungarian; (\ref{exe:kiss}a--b) exemplify identificational focus and
informational focus, respectively.


\myexe{\eenumsentence{\label{exe:kiss} 
\item\shortex{6} 
{Mari & \textsc{egy} & \textsc{kalapot} & n\'{e}zett & ki & mag\'{a}nak.}
{Mary & a & hat.\textsc{acc} & picked & out & herself.\textsc{acc}}
{`It was \textsc{a hat} that Mary picked for herself.'}
\item\shortex{6} 
{Mari & ki & n\'{e}zett & mag\'{a}nak & \textsc{egy} & \textsc{kalapot}.}
{Mary & out & picked & herself.\textsc{acc} & a & hat.\textsc{acc}}
{`Mary picked for herself \textsc{a hat}.' [hun] \citep[249]{kiss:98}}}}


\noindent According to \citeauthor{kiss:98}, (\ref{exe:kiss}a) sounds
felicitous in a situation in which Mary was trying to pick up
something at a clothing store, which implies that she chose only one
hat among the clothes in the store, and nothing
else. (\ref{exe:kiss}b), by contrast, does not presuppose such a
collection of clothes, and provides just new information that she
chose a hat. In other words, there exists an \isi{alternative set} given
within the context in (\ref{exe:kiss}a), which establishes the
difference between identificational \isi{focus} and informational focus.



\subsubsection{\citet{gundel:99}}
\label{3:sssec:gundel}


\citeauthor{gundel:99}, mainly from a semantic standpoint, divides
\isi{focus} into (c-i) psychological focus, (c-ii) semantic focus\is{semantic focus}
(also known as non-contrastive focus),\is{non-contrastive focus} and (c-iii) contrastive
focus.\is{contrastive focus} Psychological focus, according to \citeauthor{gundel:99}'s
explanation, refers to the current center of attention, and has to do
with unstressed pronouns, zero anaphora, and weakly stressed
constituents.  Among the three subtypes that \citeauthor{gundel:99}
presents, the current work takes only the last two as the subtypes of
focus, because psychological focus seems to related to information
status, rather than information structure.






\citeauthor{gundel:99} offers some differences between \isi{semantic focus}
and \isi{contrastive focus}.  First, semantic focus is the most
prosodically and/or syntactically prominent.\footnote{Of course,
  contrastive focus is also prosodically and/or morphosyntactically
  marked. What is to be noted is that semantic focus is assigned to
  the most prominent constituent in a sentence.} This is in line with
\citeauthor{givon:91}'s claim that the most important element in a
cognitive process naturally has a strong tendency to be realized in
the most marked way. This property of focus is also argued by
\citet{buring:10} as presented in \myref{def:buring:10}.\is{focus prominence}

\myexe{\enumsentence{\toplabel{def:buring:10} Focus Prominence: Focus
     needs to be maximally prominent. \citep[277]{buring:10}}}



\noindent Second, semantic \isi{focus} does not necessarily bring an entity
into psychological focus, whereas \isi{contrastive focus} always
does. Finally, semantic focus is truth-conditionally
sensitive,\is{truth-conditions} while contrastive focus has a
comparably small influence on the truth-conditions
\citep{gundel:99}.\footnote{It is reported that contrastive focus is
  sometimes relevant to the truth-conditions. Suffice it to say that
  \isi{semantic focus} is highly and necessarily sensitive to the
  truth-conditions.}




\subsubsection{\citet{gussenhoven:07}}
\label{3:sssec:gussenhoven}



\citeauthor{gussenhoven:07} classifies \isi{focus} in \ili{English} into seven
subtypes in terms of its functional usage within the context.  These
include (d-i) presentational focus, (d-ii) corrective focus, (d-iii)
counterpresupposition focus, (d-iv) definitional focus, (d-v)
contingency focus, (d-vi) reactivating focus, and (d-vii)
identificational focus.  (d-i) Presentational focus is a focused item
corresponding to \isi{\textit{wh}-words} in questions. (d-ii)
Corrective focus and (d-iii) counterpresupposition focus appear when
the speaker wants to correct an item of information that the hearer
incorrectly assumes.  In the current study, these subtypes are
regarded as \isi{contrastive focus} such that the correction test can be
used as a tool to vet contrastive focus
\citep{gryllia:09}.\is{correction test} (d-iv) Definitional focus and
(d-v) contingency focus, which usually occur with an individual-level
predicate, aim to inform the hearer of the attendant circumstances:
For example, \textit{Your \textsc{eyes} are \textsc{blue}.} states
that the eye-color of the hearer is generically blue.  (d-vi)
Reactivating focus, unlike other subtypes of focus, is assigned on
given information and is realized by the syntactic device called
focus/\isi{topic} \isi{fronting} in the present study.  Finally, (d-vii)
identificational focus \citep{kiss:98} is realized within clefts
(e.g.,\ \textit{It is \textsc{John} who she dislikes.}).\is{clefting}
The taxonomy provided by \citeauthor{gussenhoven:07} has its own
significance in that it shows the various functions that focus
performs, but the present study does not directly use it.
\citeauthor{gussenhoven:07}'s subtypes seem to be about the way in
which focus is used to different communicative ends
(i.e.\ pragmatics), which is not synonymous with focus as defined in
the current work.  Recall that I restrict the subtypes of information
structure components to those which are signaled by linguistic marking
in human languages.




\subsubsection{Summary}
\label{3:sssec:focus-sub-sum}

Regarding the subtypes of focus, the present study draws primarily
from \citet{gundel:99}, except for psychological \isi{focus} which has
more relevance to information status.  The present study classifies
focus into semantic focus (also known as non-contrastive
focus)\is{non-contrastive focus} and contrastive focus for two main
reasons.  First, in quite a few languages, these focus types are
distinctively expressed via different \isi{lexical markers} or
different positions in a clause. Second, they show clearly different
behaviors.\is{truth-conditions} In particular, \isi{semantic focus} is
relevant to truth-conditions, while contrastive focus\is{contrastive
  focus} is not so much. In contrast, \citet{lambrecht:96} provides a
classification in terms of how a sentence is configured. The
classification has to do with \isi{focus projection} and the ways in
which focus spreads from a core onto a larger phrase. The
classification that \citet{kiss:98} proposes has not been applied to
the basic taxonomy of focus in the present study, but the key
distinction (i.e.\ identificational \vs informational) is reviewed in
the analysis of cleft constructions.\is{clefting}
\citeauthor{gussenhoven:07}'s subtypes show various properties of focused
elements, but they are not also straightforwardly incorporated into
the analysis of focus herein. This is mainly because they are seldom
linguistically distinguishable.


\subsection{Linguistic properties of focus}
\label{3:ssec:properties-focus}

There are three major properties of \isi{focus} realization; (i)
\isi{inomissibility}, (ii) \isi{felicity-conditions}, and (iii)
\isi{truth-conditions}.

\subsubsection{Inomissibility}
\label{3:sssec:inomissibility}




\noindent Information structure is a matter of how information that a
speaker bears in mind is articulated in a sentence.\is{inomissibility}
Thus, formation of information structure has to do with selecting the
most efficient way to convey what a speaker wants to say.  Focus is
defined as what is new and/or important in an utterance and
necessarily refers to the most marked element in an utterance
\citep{gundel:99,buring:10}.  Due to the fact that if the maximally
prominent information is missing, a conversation becomes void and
infelicitous, focus can never be dropped from the
utterance.\footnote{Note that this distinguishes \isi{focus} as a
  component of information structure from the information status
  \textit{in focus}, since referents that are \textit{in focus} can
  often be referred to with zero anaphora \citep{gundel:03}.}  For
this reason, inomissibility has been commonly regarded as the
universal factor of focus realization in previous literature
\citep{lambrecht:96,rebuschi:tuller:99}: Only non-focused constituents
can be elided.  \citeauthor{lambrecht:96}, for instance, suggests an
ellipsis test. In \myref{exe:lambrecht:ellipsis:1}, \textit{John} and
\textit{he} convey \isi{topic} meaning, and \textit{he} can be elided
as shown in (\ref{exe:lambrecht:ellipsis:1}A2). In contrast, if the
subjects are focused, elision is disallowed as shown in
(\ref{exe:lambrecht:ellipsis:2}A2).\footnote{It appears that the
  acceptability of (\ref{exe:lambrecht:ellipsis:2}A2) differs by
  different speakers. Suffice it to say that
  (\ref{exe:lambrecht:ellipsis:2}A2) sounds less acceptable than
  (\ref{exe:lambrecht:ellipsis:2}A1).}


\myexe{\enumsentence{\toplabel{exe:lambrecht:ellipsis:1}
\begin{tabular}[t]{ll}
Q: & {What ever happened to John?}\\
A1:  & {John married \textsc{Rosa}, but he didn't really \textsc{love} her.}\\
A2:  & {John married \textsc{Rosa}, but didn't really \textsc{love} her.} \\
 & \multicolumn{1}{r}{\citep[136]{lambrecht:96}}\\
\end{tabular}}}

\myexe{\enumsentence{\toplabel{exe:lambrecht:ellipsis:2}
\begin{tabular}[t]{ll}
Q: & {Who married Rosa?}\\
A1:  & {\textsc{John} married her, but he didn't really \textsc{love} her.}\\
A2:  & {*?\textsc{John} married her, but didn't really \textsc{love} her.} \\
 & \multicolumn{1}{r}{\citep[136]{lambrecht:96}}\\
\end{tabular}}}



\noindent For this reason, the present study argues that
\myref{def:focus:mine} is the most important property of focus.\is{inomissibility}

\myexe{\enumsentence{\toplabel{def:focus:mine} Focus is an information
    structure component associated with an inomissible constituent in
    an utterance.}}


This property can also be straightforwardly applied to contrastive
focus. Constituents associated with contrastive \isi{focus} cannot be
elided, either.\footnote{In terms of the \isi{HPSG} formalism, because
  contrastive focus is also a specific type of focus, linguistic
  features plain focus involves are directly inherited into
  \isi{contrastive focus}.} This is the main distinction between
\isi{contrastive focus} and \isi{contrastive topic}. As mentioned
before, contrast is realized as either contrastive focus or
contrastive topic. In other words, a constituent conveying
\isi{contrastiveness} should be either of these. In some cases,
because many languages use the same marking system to express both
contrastive focus and contrastive topic and they share a large number
of properties, it would be hard to discriminate them using existing
tests. However, when we test whether a constituent is omissible or
not, they are distinguishable. A constituent with contrastive focus
cannot be dropped, whereas one with contrastive topic can. This
difference between them is exemplified in
Chapter~\ref{chapter5} \mypage{exe:kor:cf} with reference to
discrepancies between meanings and markings.



The fact that focus can only be assigned to constituents which are  contextually inomissible
logically entails another theorum that dropped elements
in subject/\isi{topic-drop} languages can never be evaluated as
conveying \isi{focus} meaning.  
It is well known that subjects in some languages such as 
\ili{Italian} and \ili{Spanish} can be dropped, 
which is why they are called \isi{subject-drop} languages.
What should be noted is that there is a constraint on
dropping subjects. \citet[406]{cinque:77} argues that subject
pronouns in Italian are omissible everywhere unless the subjects give
new information (i.e.\ focus from the perspective of the current
study).



I argue that \textit{pro}-drop is relevant to expressing information
structure, mainly focusing on argument optionality
\is{argument optionality} \citep{saleem:10,saleem:bender:10}: Some
languages often and optionally drop NPs with non-focus meaning. That
is, dropped arguments in \textit{pro}-drop languages must be
non-focus.  \textit{Pro}-drop can be divided into two subtypes:
\isi{subject-drop} and \isi{topic-drop}.\footnote{We cannot equate
  topics with a single grammatical category like subjects at least in
  \ili{English}, \ili{Spanish}, \ili{Korean}, and
  \ili{Chinese}. Linguistic studies, nonetheless, have provided ample
  evidence that topics and subjects have a close correlation with each
  other across languages: Subjects normally are the most unmarked
  topics in most languages
  \citep{lambrecht:96,erteschik:07}. Therefore, in more than a few
  cases, it is not easy to make a clear-cut distinction between
  subject-drop and \isi{topic-drop}, which stems from the fact that
  subjects display a tendency to be interpreted as topics.}  Typical
examples of topic-drop are shown in \myref{exe:lp:12:6} (a set of
multilingual translations, excerpted from \textit{The Little Prince}
written by \textit{Antoine de Saint-Exup{\'e}ry}). \myref{exe:lp:12:6}
are answers in \ili{English}, \ili{Spanish}, \ili{Korean}, and
(Mandarin) \ili{Chinese} to a \textit{wh}-question like \textit{What
  are you doing there?}.\is{\textit{wh}-words}


\myexe{\eenumsentence{\toplabel{exe:lp:12:6}
\item{I am drinking.}
\item\shortexnt{2}
  {\ensuremath{\emptyset} & Bebo.}
  {(I) & drink.\textsc{1sg}.\textsc{pres} [spa]}
\item\shortexnt{3}
  {\ensuremath{\emptyset} & swul & masi-n-at.}
  {(I) & alcohol & drink-\textsc{pres-decl} [kor]}
\item\shortexnt{3}
  {(w\v{o}) & h\={e} & ji\v{u}.}
  {I & drink & alcohol [cmn]}}}

\noindent The subjects in (\ref{exe:lp:12:6}a--d) are all first person,
and also function as the \isi{topic} of the sentences. The different
languages have several different characteristics.  (i) The use of `I'
in (\ref{exe:lp:12:6}a) is obligatory in \ili{English}.  (ii) The
subject in Spanish, a morphologically rich language, can be freely
dropped as shown in (\ref{exe:lp:12:6}b). (iii) The subject in
\ili{Korean} is also highly omissible as shown in
(\ref{exe:lp:12:6}c), though Korean does not employ any agreement in
the morphological paradigm. (iv) Chinese, like English, is
morphologically impoverished, and also is like Korean in that it does
not have inflection on the verb according to the subject. The subject
in Chinese (e.g.\ \textit{w\v{o}} `I' in \ref{exe:lp:12:6}d) can be
dropped as well.  The subjectless sentences exemplified in
(\ref{exe:lp:12:6}c--d) in Korean and Chinese have been regarded as
instances of \isi{topic-drop} in quite a few previous studies in
tandem with the subjectless sentences in \isi{subject-drop} languages
(e.g.\ \ili{Spanish})
\citep{li:thompson:76,huang:84,yang:02,alonso:etal:02}.


\subsubsection{Felicity-conditions}
\label{3:sssec:truth-condition:felicity-condition}



Felicity is conditioned by how a speaker organizes an utterance with
respect to a particular context.\is{felicity} For this reason,
information structure generally affects felicity
conditions.\is{felicity-conditions} That is, information structure
should be interpreted with respect to the contexts in which an
utterance of a particular form can be successfully and cooperatively
used.


Information structure has often been studied in terms of
allosentences. These are close paraphrases which share
truth-conditions \citep{lambrecht:96}.  \citet{engdahl:vallduvi:96}
begin their analysis with a set of allosentences though they do not
use that terminology. Allosentences
(\ref{exe:engdahl:vallduvi:96:2}a--b) differ in the way their content
is packaged: (\ref{exe:engdahl:vallduvi:96:2}a) in which the object is
focused is an appropriate answer to a question like \textit{What does
  he hate?}, while (\ref{exe:engdahl:vallduvi:96:2}b) in which the
verb is focused is not. Propositions in
(\ref{exe:engdahl:vallduvi:96:2}a--b) have in common what they assert
about the world (i.e.\ the same truth-condition), but differ in the
way the given information is structured.

\myexe{\eenumsentence{\toplabel{exe:engdahl:vallduvi:96:2}
\item He hates \textsc{chocolate}.
\item He \textsc{hates} chocolate.
\item Chocolate he \textsc{loves}. \citep[2]{engdahl:vallduvi:96}}}

\noindent In a nutshell, allosentences are sentences which differ only
in felicity-conditions. Although a set of allosentences is comprised
of exactly the same propositional content, the sentences convey
different meanings from each other, and the differences are caused by
how \isi{focus} is differently expressed.




\subsubsection{Truth-conditions}
\label{3:sssec:focus-sensitivity}

Information structure can also impact truth conditions
\citep{partee:91,gundel:99}.\is{truth-conditions}\is{focus sensitive
  item} \citet{beaver:clark:08} claim that focus sensitive items
deliver complex and non-trivial meanings which differ from language to
language, and their contribution to meaning is rather difficult to
elicit.  What is notable with respect to \isi{focus} sensitive items
is that if there is an item whose contribution to the
truth-conditional semantics (e.g.,\ where it attaches in the semantic
structure) is focus-sensitive, then changes in information structure
over what is otherwise the same sentence containing that item should
correlate with changes in truth-conditions.


Focus sensitive items related to truth-conditions include modal verbs
(e.g.\ \textit{must}), frequency adverbs (e.g.\ \textit{always}),
counterfactuals, \isi{focus} particles (e.g.\ \textit{only},
\textit{also}, and \textit{even}), and superlatives
(e.g.\ \textit{first}, \textit{most}, etc.)  \citep{partee:91}.  One
well known example is shown in (\ref{exe:dogs-must-be-carried}),
originally taken from \citet{halliday:70}.


\myexe{\eenumsentence{\label{exe:dogs-must-be-carried}
\item{Dogs must be \textsc{carried}.}
\item{\textsc{Dogs} must be carried. \citep[169]{partee:91}}}}


\noindent They are respectively interpreted as (a)
MUST(dog(\textit{x}) \& here(\textit{x}), \textit{x} is carried) and
(b) MUST(here(\textit{e}), a dog or dogs is/are carried at \textit{e})
or MUST(you carry \textit{x} here, you carry a dog here)
\citep[169]{partee:91}.  In other words, the focused items in
(\ref{exe:dogs-must-be-carried}a--b) differ, and they cause
differences in truth-conditions.\is{truth-conditions} To take another
example, the sentences shown in (\ref{exe:exam}b--c) do not share the
same truth-conditions due to two focus-sensitive operators
\textit{most} and \textit{first}.  They convey different meanings
depending on which item the A-accent (H* in the ToBI format,
\citealt{bolinger:61,jackendoff:72}) falls
on.\is{A-accent}\is{truth-conditions}

\myexe{\eenumsentence{\toplabel{exe:exam}
\item The most students got \textsc{between} 80 \textsc{and} 90 on the
  first quiz.
\item The most students got between 80 and 90 on the \textsc{first}
  quiz. \citep[172]{partee:91}}}




\subsection{Tests for Focus}
\label{3:ssec:tests-focus}

As exemplified several times so far, \textit{wh}-questions are
commonly used to probe meaning and marking of \isi{focus}
\citep{lambrecht:96,gundel:99}.\is{\textit{wh}-questions} The phrase
answering the \textit{wh}-word of the question is focused in most
cases; the focused part of the reply may be either a word
(i.e.\ \isi{narrow focus}, or argument focus), a phrase consisting of
multiple words (i.e.\ \isi{wide focus}, or predicate focus), or a
sentence including the focused item (i.e.\ all focus, or sentence
focus).  For instance, if a \textit{wh}-question is given like
\textit{What barks?}, the corresponding answer to the \textit{wh}-word
bears the A-accent, such as \textit{The \textsc{dog}
  barks.}\is{A-accent}


It seems clear that using \textit{wh}-questions is a very reliable
test for identifying \isi{focus} in the sense that we can determine which
linguistic means are used in a language. Yet, there are also instances
in which \textit{wh}-questions cannot be used. In particular, it is
sometimes problematic to use \textit{wh}-questions to locate focused
constituents in running texts, which do not necessarily consist of Q/A
pairs.  Moreover, it can be difficult to pinpoint focused elements
unless the marking system used is orthographically expressed. For
instance, since the primary way to express information structure
meanings in \ili{English} is
prosody,\is{\textit{wh}-questions}\is{prosody} \textit{wh}-questions
are unlikely to be determinate when analyzing written texts in
English.\footnote{This problem is also raised by \citet{gracheva:13},
  who utilizes the Russian National Corpus for a study of contrastive
  structures in \ili{Russian}.\is{contrast} She points out that it is troublesome
  to apply existing tests of information structure, such as
  \textit{wh}-questions, to naturally occurring speech. This is
  because, when working with running text, it is actually impossible
  to separate a single sentence from the context and test it
  independently.}



In order to make up for the potential shortcomings of the
\textit{wh}-test, the present study employs the \isi{deletion test},
leveraging the fact that focused items are inomissible. As illustrated
in the previous subsection (Section \ref{3:sssec:inomissibility}),
\isi{inomissibility} is an essential linguistic property of
focus.\footnote{Another test for \isi{focus} is identifying the strongest
  stress \citep{rebuschi:tuller:99}, but \citet{casielles:04} provides
  a counterexample to this test. \citeauthor{casielles:04} reveals
  that primary stress does not always guarantee the focus even in
  \ili{English}. In particular, finding the more remarkable stress is
  not available for the present study that basically aims at text
  processing.}





\section{Topic}
\label{3:sec:topic}


\subsection{Definition}
\label{3:ssec:definition-topic}


Topic refers to what a sentence is about
\citep{strawson:64,lambrecht:96,choi:99}, which can be defined as
\myref{def:topic}.

\myexe{\enumsentence{\toplabel{def:topic} An entity, E, is the topic
    of a sentence, S, iff in using S the speaker intends to increase
    the addressee's knowledge about, request information about, or
    otherwise get the addressee to act with respect to
    E. \citep[210]{gundel:88}}}

There is an opposing point of view to this generalization.
\citet{vermeulen:09} argues that what the sentence is about is not
necessarily the \isi{topic} of the sentence.  \citeauthor{vermeulen:09} does
not analyze the subject \textit{he} in (\ref{exe:vermeulen:338}A) as
the topic of, even though the sentence is about the subject.
According to \citeauthor{vermeulen:09}, \textit{he} is an anaphoric
item that merely refers back to the so-called discourse topic
\textit{Max} in (\ref{exe:vermeulen:338}Q).

\myexe{\enumsentence{\label{exe:vermeulen:338}
\begin{tabular}[t]{ll}
Q: & {Who did Max see yesterday?}\\
A: & {He saw Rosa yesterday.}\\
\end{tabular}}}



\noindent However, this analysis is not adopted by the current work
for two reasons: First, \citeauthor{vermeulen:09}'s argument runs
counter to the basic assumption presented by \citet{lambrecht:96}, who
asserts that a topic has to designate a discourse referent internal to
the given context. Second, if the answer (\ref{exe:vermeulen:338}A),
given to a question like \textit{Who did Max see yesterday?}, is
translated into \ili{Korean} in which the \nun marker is used in
complementary distribution with the nominative marker \ika,
\textit{ku} `he' can be combined with only the \nun marker as shown in
(\ref{exe:vermeulen:338}$^\ensuremath{\prime}$).

\myexe{\enumsentence[(\ref{exe:vermeulen:338}$^\ensuremath{\prime}$)]{
\begin{tabular}[t]{ll}
Q: & { \shortex{4}{Mayksu-nun/ka & ecey & mwues-ul
    & po-ass-ni?}  {Max-\textsc{nun}/\textsc{nom} & yesterday &
    what-\textsc{acc} & see-\textsc{pst}-\textsc{ques}} {`What did Max
    see yesterday.'}}\\
A: & {\shortex{4}{ku-nun/\#ka & ecey & losa-lul & po-ass-e.}
  {he-\textsc{nun}/\textsc{nom} & yesterday & Rosa-\textsc{acc} &
    see-\textsc{pst}-\textsc{decl}} {`He saw Rosa yesterday.'
    [kor]}}\\
\end{tabular}}}


\noindent That is to say, though \textit{he}
in (\ref{exe:vermeulen:338}A) is an anaphoric element connecting to the
discourse \isi{topic} \textit{Max}, it can function as the topic in at least
one language with relatively clear marking of topic.  



There is another question partially related to \myref{def:topic}: Are
there topicless sentences?\is{topicless}  There are two different viewpoints on
this.  \citet{erteschik:07} argues that every sentence has a \isi{topic},
though the topic does not overtly appear, and a topic that covertly
exists is a so-called stage topic.  According to
\citeauthor{erteschik:07}'s claim, topic is always given in sentences
in human language, because topic is relevant to knowledge the hearer
possesses. In contrast, \citet{buring:99} argues that topic may be
non-existent, in terms of sentential forms. \citeauthor{buring:99}
assumes that sentences, in terms of information structure, are
composed of \isi{focus}, topic, and \isi{background}, and a sentence may be
either an all-focus construction, a bipartite construction
(i.e.\ lacking topic), or a tripartite construction consisting of all
three components, including \isi{background}.  In fact, these 
arguments are not incompatible with \citeauthor{erteschik:07} simply
putting more emphasis on psychological status, and
\citeauthor{buring:99} emphasizing form. The present study follows
\citeauthor{buring:99}'s argument, because I am interested in mapping
linguistic forms to information structure meaning.





\subsection{Subtypes of topic}
\label{3:ssec:subtypes-topic}


Given that \isi{contrast} is one of the cross-cutting categories in
information structure, topics can be divided into two subtypes:
\isi{contrastive topic} and \isi{non-contrastive topic} (here renamed
aboutness-topic\is{aboutness topic} in line with \citeauthor{choi:99}'s claim that
aboutness is the core concept of regular topics).  In comparison with
other components of information structure, contrastive topic has been
relatively understudied, with a few notable exceptions.  Contrastive
topic has been addressed in \ili{Japanese} and \ili{Korean} in
reference to \wa and \nun (also known as topic markers)
\citep{kuno:73,choi:99}. Additionally, \citet{arregi:03} identifies
clitic \isi{left dislocation} in \ili{Spanish} and other languages as
a syntactic operation to articulate contrastive topic.


In addition to those outlined above, \citet{fery:krifka:08} present
another subtype of topics: \isi{frame-setting} topics.
\cite[50]{chafe:76} defines a frame-setting topic as an element
which sets ``a spatial, temporal or individual framework within which
the main predication holds'', and it can be formally defined as
(\ref{def:jacobs}).

\myexe{\enumsentence{\toplabel{def:jacobs} Frame-setting: In (X Y), X is
    the frame for Y iff X specifies a domain of (possible) reality to
    which the proposition expressed by Y is
    restricted. \citep[656]{jacobs:01}}}


\noindent This terminology is not directly included into the taxonomy
of information structure meanings (i.e.\ the type hierarchy of
information structure) in the current work, but its linguistic
constraints are incorporated into the information structure
library. This is mainly because \isi{frame-setting} topics are
redundant with other \isi{topic} types (particularly, \isi{contrastive topic})
with respect to semantic representation.


Frame-setting topics are universally associated with
sentence-initial adjuncts \citep[118]{lambrecht:96} though not all
sentence-initial adjuncts are necessarily frame-setting topics
(i.e.,\ the relation is not bidirectional).\is{frame-setting} In other
words, frame-setting topics have one constraint on sentence
positioning; they should be sen\-tence-initial.  \citet{fery:krifka:08}
give an example of frame setting as shown in
(\ref{exe:fery:krifka:128}), in which the sentence talks about the
subject \textit{John}, but is only concerned with his health. Thus,
the aboutness \isi{topic} is assigned to \textit{John}, but frame setting
narrows down the aspect of description.\footnote{In a similar vein but
  from a different point of view, \citet[656]{jacobs:01} argues
  that frame-setting is divergent from the other topics in terms of
  dimensions of Topic-Comment: ``..., frame-setting is not a feature
  of all instances of TC but just one of the dimensions of TC that may
  or may not occur in TC sentences.''  }


\myexe{\enumsentence{\toplabel{exe:fery:krifka:128}
\begin{tabular}[t]{ll}
Q: & {How is John?}\\
A: & {\{Healthwise\}, he is \textsc{fine} \citep[128]{fery:krifka:08}.}\\
\end{tabular}}}










As mentioned above, the present work does not regard frame-setting
topics as one of the cross-cutting components that semantically
contribute to information structure.  Since the property of
frame-setting topics refers to a syntactic operation expressing
information structure, hereafter it is referred to as a frame-setter.
Various types of constructions can be used as a frame-setter.  First,
the `as for ...'  construction in \ili{English} serves as a frame-setter as
exemplified in \myref{exe:fery:krifka:128} and
\myref{exe:fery:krifka:128:2}.



\myexe{\enumsentence{\toplabel{exe:fery:krifka:128:2}
\begin{tabular}[t]{ll}
Q: & {How is John?}\\
A: & {\{As for his health\}, he is \textsc{fine} \citep[128]{fery:krifka:08}.}\\
\end{tabular}}}

\noindent Second, adverbial categories (e.g.\ \textit{Healthwise} in
\ref{exe:fery:krifka:128}A) can sometimes serve as frame-setters.
For instance, regarding (i), in \myref{exe:dipper:etal:fs} in
\ili{German}, where \textit{gestern abend} `yesterday evening' and
\textit{k\"orperlich} `physically' are fronted, the frame-setting
\is{topic}to\-pic is assigned to the adverbials.  This property is conjectured to
be applicable to all other languages \citep{chafe:76,lambrecht:96}.


\myexe{\eenumsentence{\toplabel{exe:dipper:etal:fs}
\item\shortex{7}
{Gestern & abend & haben & wir & Skat & gespielt.}
{yesterday & evening & have & we & Skat & played}
{`Yesterday evening, we played Skat.'}
\item\shortex{6}
{K\"orperlich & geht & es & Peter & sehr & gut.}
{physically & goes & it & Peter & very & well}
{`Physically, Peter is doing very well.' [ger] \citep[169]{dipper:etal:07}}}}


\noindent An adjunct NP can sometimes be a frame-setter, if and only
if it appears in the sentence-initial position. In the \ili{Japanese}
sentence \myref{exe:jpn:frame-setting} below, the genuine subject of
the sentence is \textit{supiido suketaa} `speed skater', while
\textit{Amerika} `America' restricts the domain of what the speaker is
talking about. Note that since frame-setters are realized as a topic,
they are normally realized by the
\wa-marking.\is{frame-setting}\is{frame-setter}


\myexe{\enumsentence{\toplabel{exe:jpn:frame-setting}
\shortex{6}
{Amerika & wa & supiido & suketaa & ga & hayai.}
{America & \textsc{wa} & speed & skater & \textsc{nom} & fast}
{`As for America, the speed skaters are fast.' [jpn]}}}

\noindent Therefore, the first NP combined with \wa is interpreted as
\isi{topic}, whereas the second phrase with the nominative marker \ga, which
functions as the subject, does not convey topic meaning.  Adjunct
clauses which set the frame to restrict the temporal or spatial
situation of the current discourse also have a topic relation with the
main clause.  \citet{haiman:78} and \citet{ramsay:87} argue that
sentence-initial conditional clauses function as the topic of the
whole utterance.  The same goes for sentence-initial temporal
clauses. For instance, in \myref{lp:eng:14:13}, taken from
the translation of \textit{The
  Little Prince} written by \textit{Antoine de Saint-Exup{\'e}ry}, the
entire temporal clause \textit{when he arrived on the planet} is dealt
with as the frame-setter of the sentence.

\myexe{\enumsentence{\label{lp:eng:14:13}
When he arrived on the planet, he respectfully saluted the lamplighter.}}



In sum, frame-setters must show up before anything else, and this
holds true presumably across all languages
\citep{chafe:76,lambrecht:96}. The role of frame-setters is assigned
to sentence-initial constituents which narrow down the domain of what
the speaker is talking about as defined in (\ref{def:jacobs}).  It can
be assigned to various types of phrases including adverbs and even
adjunct clauses.  The syntactic restrictions on frame-setters are not
reflected in the current classification of topics, because the present
work intends to provide a semantics-based classification of
information structure components.  The only semantic distinctions
between frame-setter and other topics are orthogonal to information
structure.\is{frame-setter}



\subsection{Linguistic properties of topic}
\label{3:ssec:properties-topic}



This subsection discusses several linguistic properties which should
be taken into consideration in the creation of a computational model
for the realization of topics; (i) scopal interpretation relying on
the \isi{topic} relation, (ii) clausal constraints, (iii) multiple
topics, and (iv) verbal topics.



\subsubsection{Scopal interpretation}
\label{3:sssec:scopal-interpretation}

Many previous studies argue that topics take wide scope. For instance,
according to \citet{buring:97}, if a rise-fall accent contour in
\ili{German} co-occurs with negation,\is{negation} the prosodic
marking disambiguates a scopal interpretation. For example,
(\ref{exe:buring:scope}a) would have two scopal readings if it were
not for prosodic marking, but in (\ref{exe:buring:scope}b), in which
`\ensuremath{\slash}' and `\ensuremath{\backslash}' stand for rise and
fall respectively, there is only a single available meaning.


\myexe{\eenumsentence{\toplabel{exe:buring:scope}
\item\shortex{5}
  {Alle & Politiker & sind & nicht & korrupt.}
  {all & politicians & are & not & corrupt}
  { (a) \ensuremath{\surd}\ensuremath{\forall}\ensuremath{>}\ensuremath{\neg} `For all politicians, it is not the case that they are corrupt.' \\
    (b) \ensuremath{\surd}\ensuremath{\neg}\ensuremath{>}\ensuremath{\forall} `It is not the case that all politicians are corrupt.'}
\item\shortex{5}
  {\ensuremath{\slash} \textsc{Alle} & Politiker & sind & \textsc{nicht} \ensuremath{\backslash} & korrupt.}
  {all & politicians & are & not & corrupt}
  { *\ensuremath{\forall}\ensuremath{>}\ensuremath{\neg}, \ensuremath{\surd}\ensuremath{\neg}\ensuremath{>}\ensuremath{\forall} [ger] \citep[175]{buring:97}}}}






\subsubsection{Clausal constraints}
\label{3:sssec:clausal-constraints}



Topics can appear in non-matrix clauses, but there are some clausal
constraints.  \citet[126]{lambrecht:96} offers an observation that
some languages mark the difference in topicality between matrix and
non-matrix clauses by morphosyntactic means.  To take a well known
example, \citet{kuno:73} argues that the \isi{topic} marker \wa in
\ili{Japanese} tends not to be attached to NPs in embedded clauses,
and it is my intuition that \ili{Korean} shares the same tendency.
Yet, a tendency is just a tendency.  Some subordinate clauses are
evaluated as containing topic.  \myref{exe:lim:14} presents two
counterexamples from Korean.


\myexe{\eenumsentence{\toplabel{exe:lim:14}
\item\shortex{4}
  {hyangki-nun & coh-un & kkoch-i & phi-n-ta.}
  {scent-\textsc{nun} & good-\textsc{rel} & flower-\textsc{nom} & bloom-\textsc{pres}-\textsc{decl}}
  {`A flower with a good scent blooms.'}
\item\shortex{4}
  {Chelswuka-ka & insayng-un & yuhanha-tako & malha-yss-ta.}
  {Cheolsoo-\textsc{nom} & life-\textsc{nun} & limited-\textsc{comp} & say-\textsc{pst}-\textsc{decl}}
  {`Cheolsoo said that life is limited.' [kor] \citep[229]{lim:12}}}}



\noindent First, \citeauthor{lim:12} argues that \nun can be used in a
\isi{relative clause} as given in (\ref{exe:lim:14}a) when the
\onun-marked NP conveys a contrastive meaning (i.e.\ a contrastive
\isi{focus} in this case).  The relative clause in (\ref{exe:lim:14}a),
which modifies the following NP \textit{kkoch} `flower', conveys a
meaning like \textit{The flower smells good, but contrastively it does
  not look so good.}, and \nun attached to \textit{hyangki} `scent' is
responsible for the contrastive reading. If \textit{hyangki} is
combined with a nominative marker \textit{ka}, instead of
\textit{nun}, the sentence still sounds good as presented in
(\ref{exe:lim:14}$^\ensuremath{\prime}$a), but the contrastive meaning
becomes very weak or just disappears.


\myexe{\eenumsentence[\ref{exe:lim:14}$^\ensuremath{\prime}$]{
\item\shortex{4}
  {hyangki-ka & coh-un & kkoch-i & phi-n-ta.}
  {scent-\textsc{nom} & good-\textsc{rel} & flower-\textsc{nom} & bloom-\textsc{pres}-\textsc{decl}}
  {`A flower with a good scent blooms.' [kor]}}}


\noindent Second, if the main predicate is concerned with speech acts
(e.g.\ \textit{malha} `say') as is the case in (\ref{exe:lim:14}b),
non-matrix clauses can have topicalized constituents.


The relationship between \isi{topic} and clausal types has been discussed in
previous literature with special attention to the so-called root
phenomena
\citep{haegeman:04,heycock:07,bianchi:frascarelli:10,roberts:11}.
\citeauthor{roberts:11}, for instance, provides several \ili{English}
examples in which the topic shows up in the non-matrix clauses.\is{root phenomena}

\myexe{\eenumsentence{\label{exe:roberts:embedded-topic}
\item Bill warned us that \myemp{flights to Chicago} we should try to
  avoid. \citep[77]{emonds:04}
\item It appears that \myemp{this book} he read
  thoroughly. \citep[478]{hooper:thompson:73}
\item I am glad that \myemp{this unrewarding job}, she has finally
  decided to give up. \citep[69]{bianchi:frascarelli:10}}}


\noindent These examples imply that left-dislocated constituents can
appear in embedded clauses even in \ili{English}, if the main
predicate denotes speech acts (e.g.\ \textit{warn} in
\ref{exe:roberts:embedded-topic}a), quasi-evidentials
(e.g.\ \textit{it appears} in \ref{exe:roberts:embedded-topic}b), or
(semi-)factives (e.g.\ \textit{be glad} in
\ref{exe:roberts:embedded-topic}c).



This property of topics in non-matrix clauses needs to be considered
when I build up a model of information structure for multiclausal
utterances. The relevant constraints are reexamined in
Chapter~\ref{chapter10-2} in detail.




\subsubsection{Multiple topics}
\label{3:sssec:multiple-topics}

\citet{bianchi:frascarelli:10} argue that aboutness topics (A-Topics
in their terminology)\is{aboutness topic} can appear only once, while other types of
topics, such as contrastive topics (C-Topics),\is{contrastive topic} can turn up multiple
times.\is{aboutness topic}  The present study looks at the difference in terms of
discrepancies between marking and meaning of information structure.
As discussed previously at the beginning of this chapter, \isi{topic}-marked
elements may or may not occur in a single clause. Notably, they can
appear multiple times as exemplified in
\myref{exe:multiple:topics:kor}.

\myexe{\enumsentence{\label{exe:multiple:topics:kor}
\shortex{3}
{Kim-un & chayk-un & ilk-ess-ta.}
{Kim-\textsc{nun} & book-\textsc{nun} & read-\textsc{pst}-\textsc{decl}}
{`Kim read the book.' [kor]}}}

\noindent However, the \onun-marked NPs in
\myref{exe:multiple:topics:kor} do not carry the same status with
respect to information structure. The \onun-marked subject
\textit{Kim-un} in situ in \myref{exe:multiple:topics:kor}
can be either an \isi{aboutness topic} or a \isi{contrastive topic}, because
\ili{Korean} is a topic-first language \citep{sohn:01}. In contrast,
the \onun-marked object \textit{chayk-un} has a contrastive meaning,
because \onun-marked non-subjects in situ are normally
associated with contrastive \isi{focus} \citep{choi:99,song:bender:11}. In
short, \onun-marked (i.e.\ topic-marked) constituents can occur
multiple times, but not all of them are necessarily associated with
topic.  Thus, topic marker is not an appropriate name at least for
\nun in Korean (and \wa in Japanese). Section \ref{5:sec:lex} provides more
discussion on meanings that \nun and \wa-marked items in Korean and
Japanese carry.





\subsubsection{Verbal topics}
\label{3:sssec:verbal-topics}


Topic-marking on verbal items is rare, but a cross-linguistic survey
of information structure markings provides one exceptional case:
\ili{Paumar{\'{\i}}} does employ a verbal topic marker, which cannot
co-occur with a nominal \isi{topic} marker in the language
\citep{chapman:81}. Therefore, the present study assumes that topic
can be assigned to verbal items, and the possibility is one of the
language-specific parameters that needs to be considered when I
describe and implement the web-based questionnaire
(Section \ref{11:sec:questionnaire}). In the questionnaire, I let users choose
a categorical constraint on \isi{focus} and topic. Although topics are
normally assigned to NPs, users are able to choose verbal markings of
topic.




\subsection{Tests for topic}
\label{3:ssec:tests-topic}

Given that the treatment of aboutness as the semantic core of topics
is supported by many previous studies, \citet{reinhart:81} and
\citet{choi:99} suggest a diagnostic to identify \isi{topic}, namely the
\textit{tell-me-about} test.\is{\textit{tell-me-about} test} For
instance, a reply to \textit{Tell me about the dog} will contain a
word with the B-accent (L+H*) in \ili{English}, such as \textit{The
  \textbf{dog} \textsc{barks}.}  This test can be validly used across
languages.\is{B-accent} For example in \ili{Korean}, the word that
serves as the key answer to \textit{tell-me-about} must not be
realized with case markers that have the non-topic relation, as
exemplified in (\ref{exe:kor:tell-me-about}).


\myexe{\eenumsentence{\toplabel{exe:kor:tell-me-about}
\item[Q:]\shortex{5}
{ku & kay-ey & tayhayse & malha-y & cwu-e.}
{the & dog-\textsc{dat} & about & talk-\textsc{comp} & give-\textsc{imp}}
{`Tell me about the dog.'}
\item[A:]\shortex{4}
{ku & kay-\#ka/nun & cacwu & cic-e}
{the & dog-\textsc{nom}/\textsc{nun} & often & bark-\textsc{decl}}
{`The \textbf{dog} often barks.' [kor]}}}


Nonetheless, there are a few opposing claims \citep{vermeulen:09}, and
several additional tests have been devised that take notice of the
relationship between topichood and aboutness.  \citet{roberts:11}, in
line with \citet{reinhart:81} and \citet{gundel:85}, provides four
paraphrasing tests for topic in \ili{English} as follows, which differ
subtly in their felicity-conditions.\is{felicity-conditions} If a
left-dislocated NP conveys a meaning of topic, the constituent can be
paraphrased as at least one of the constructions presented below.


\myexe{\eenumsentence{\toplabel{exe:roberts:topic-test}
\item{\myemp{About} Coppola, he said that he found him to be ...}
\item{\myemp{What about} Coppola? He found him to be ...}
\item{\myemp{As for} Coppola, he found him to be ...}
\item{\myemp{Speaking of} Coppola, he found him to be ... \citep[1916]{roberts:11}}}}


\noindent As \citet{roberts:11} explains, those tests may not be
straightforwardly applicable to other languages because translations
can vary in accordance with fairly delicate differences in felicity.
\citet{oshima:09} suggests using the \textit{as-for} test (which can
be translated into \textit{ni-tsuite-wa}) to test for topic in
\ili{Japanese}. The test can be defined as \myref{def:as-for} with
examples in Japanese given in \myref{exe:oshima:as-for}. For example,
\textit{Ken-wa} in (\ref{exe:oshima:as-for}a) and \textit{Iriasu-wa}
(\ref{exe:oshima:as-for}c) are evaluated as containing \isi{topic} meaning
because they pass the \textit{as-for} test.

\myexe{\enumsentence{\toplabel{def:as-for} The \textit{as for} test:
    If an utterance of the form: [$_{\textnormal{\tiny{S}$_{1}$}}$
      ... X ...] can be felicitously paraphrased as [\textit{As for}
      X, S\mysub{2}] where S\mysub{2} is identical to S\mysub{1}
    except that X is replaced by a pronominal or empty form anaphoric
    to X, X in S\mysub{1} is a topic. \citep[410]{oshima:09}}}



\myexe{\eenumsentence{\toplabel{exe:oshima:as-for}
\item\shortex{3}
{Ken-wa & Iriasu-o & yomi-mashi-ta.}
{Ken-\textsc{wa} & Iliad-\textsc{acc} & read-\textsc{polite}-\textsc{pst}}
{`Ken read Iliad.'  [jpn]}
\item\shortex{3}
{Ken-ni-tsuite-wa, & Iriasu-o & yomi-mashi-ta.}
{Ken-ni-tsuite-\textsc{wa} & Iliad-\textsc{acc} & read-\textsc{polite}-\textsc{pst}}
{`As for Ken, he read Iliad.'  [jpn]}
\item\shortex{3}
{Iriasu-wa & Ken-ga & yomi-mashi-ta.}
{Iliad-\textsc{wa} & Ken-\textsc{nom} & read-\textsc{polite}-\textsc{pst}}
{`As for Iliad, Ken read it.' [jpn]}
\item\shortex{3}
{Iriasu-ni-tsuite-wa, & Ken-ga & yomi-mashi-ta.}
{Iliad-ni-tsuite-\textsc{wa} & Ken-\textsc{nom} & read-\textsc{polite}-\textsc{pst}}
{`As for Iliad, Ken read it.' [jpn] \citep[410]{oshima:09}}}}





Given that aboutness is the semantic core of \isi{topic} from a
cross-linguistic view, the current work employs the diagnostics
presented by \citet{roberts:11} as exemplified in
\myref{exe:roberts:topic-test}.  In the current work, topic is assumed
to be assigned to an entity that can pass one of the paraphrasing
tests in \citeauthor{roberts:11}.









\section{Contrast}
\label{3:sec:contrast}


\subsection{Definition}
\label{3:ssec:definition-contrast}


In the present work, contrast is treated as a cross-cutting
information structure component, which contributes the entailment of
an \isi{alternative set} \citep{molnar:02,krifka:08}.  Since contrast
can never show up out of the blue \citep[9]{erteschik:07}, the
existence of an \isi{alternative set} within the discourse is essential for
contrast.\is{contrast} 


A so-called alternative set which posits \isi{focus} semantic values is
suggested within the framework of alternative semantics
(\citealt{rooth:85,rooth:92}).  What follows briefly shows how focus
alternatives are calculated.\is{alternative set} In line with
\citeauthor{rooth:92}'s proposal, a sentence \ensuremath{S} has three
semantic values; the ordinary value \ensuremath{\llbracket S
  \rrbracket}$^{o}$, the focus value \ensuremath{\llbracket S
  \rrbracket}$^{f}$, plus the \isi{topic} value \ensuremath{\llbracket S
  \rrbracket}$^{t}$.  The ordinary value is a proposition, and the
focus value is a set of propositions, and the topic value is a set of
sets of propositions \citep[184]{nakanishi:07}.  For example,
given that a discourse \ensuremath{D} involves an ontology
\ensuremath{D} consisting of \{John, Bill, David, Sue\}, the semantic
values with respect to (\ref{exe:alternative-set}b) are composed of
elements in the alternative set (\ref{exe:alternative-set}c). Note
that the element in the \isi{focus} domain (i.e.\ \textit{Bill}) is altered
into other elements in the ontology.



\myexe{\eenumsentence{\label{exe:alternative-set} 
\item{Who did John introduce to Sue?}
\item{\textbf{John} introduced [$_{f}$ \textsc{Bill}] to Sue.}
\item{\ensuremath{\llbracket (\textnormal{\ref{exe:alternative-set}a}) \rrbracket}$^{o}$ = who did John introduce to Sue}
\item{\ensuremath{\llbracket (\textnormal{\ref{exe:alternative-set}b}) \rrbracket}$^{f}$ = \{ [John introduced John to Sue], [John introduced Bill to
  Sue], [John introduced David to Sue], [John introduced John to Sue]
  \}}
\item{\ensuremath{\llbracket (\textnormal{\ref{exe:alternative-set}b}) \rrbracket}$^{t}$ = \{ 
\{ [John introduced John to Sue], [John introduced Bill to Sue], [John introduced David to Sue], [John introduced John to Sue] \},
\{ [Bill introduced John to Sue], [Bill introduced Bill to Sue], [Bill introduced David to Sue], [Bill introduced John to Sue] \},
... \}}}}


\noindent If the \isi{alternative set} is invoked in the given discourse
with an exclusive meaning, we can say \textit{Bill} in
(\ref{exe:alternative-set}b) is 
contrastively focused.\is{contrastive focus} That is, the
\isi{focus} value and the topic value define the alternative set, with
respect to an ontology \ensuremath{D}.

This notion is actually very similar to (or even the same as) the
so-called trivialization set proposed by
\citet{buring:99}.\is{\textit{wh}-questions} It can be formulated as
the following rule relating a \textit{wh}-question to its
corresponding reply, given that \ensuremath{\cup\llbracket S
  \rrbracket}$^{f}$ is as informative as the solicited question.

\myexe{\enumsentence{\label{formula:buring:99} A sentence \ensuremath{A} can be
    appropriately uttered as an answer to a question \ensuremath{Q} iff
    \ensuremath{\cup\llbracket A \rrbracket}$^{f}$=\ensuremath{\cup\llbracket Q \rrbracket}$^{o}$}}

\noindent Building upon \myref{formula:buring:99}, the following Q/A
pairs are ill-formed. The question in the second pair, for instance,
does not presuppose the pop stars wore caftans, but the answer does
with narrowly focusing on the color of caftans.\is{narrow focus}


\myexe{\enumsentence{\toplabel{exe:buring:99} 
\begin{tabular}[t]{ll}
Q: & {What kind of caftans did the pop stars wear?}\\
A: & {\#All the pop stars wore [$_{f}$ dark \textsc{caftans}].}\\
Q: & {What did the pop stars wear?}\\
A: & {\#All the pop stars wore [$_{f}$ \textsc{dark}] caftans. \citep[144]{buring:99}} \\
\end{tabular}}}



\noindent Turning back to the alternative set, \textsc{caftan} and
\textsc{dark} with the A-accent in (\ref{exe:buring:99}a--b)
respectively are not included in the alternative set of the given
discourse.\is{A-accent} Thus, they can be focused neither
non-contrastively nor contrastively, because they do not invoke any
\isi{alternative set}. In sum, the existence of an alternative set is
essential for articulating the information structural meaning of
contrast.\is{contrast}








As with non-contrastive \isi{topic} and \isi{focus}, contrast may be marked by
virtually any linguistic means (prosodic, lexical, and/or syntactic)
across languages, and the same device may mark both non-contrastive
and contrastive constituents.\is{contrast}  For example, \citet[296]{gundel:99}
argues that placing a constituent in a specific sentence position
(e.g.\ the sentence-initial position in \ili{English}) can be used to mark
either \isi{non-contrastive focus} or purely \isi{contrastive focus}.  Topic can
also have a contrastive meaning, and sometimes non-contrastive topic
and contrastive topic share the same linguistic means.  For example,
\ili{Korean} which employs \nun can express contrastive topic as shown
in \myref{exe:kor:contrastive-topic}; the answer conveys an
interpretation like \textit{I surely know Kim read a book, but I think
  Lee, contrastively, might not have.}

\myexe{\eenumsentence{\label{exe:kor:contrastive-topic}
\item[Q:]\shortex{4}
{Kim-kwa & Lee-nun & mwuess-ul & ilk-ess-ni?}
{Kim-and & Lee-\textsc{nun} & what-\textsc{acc} & read-\textsc{pst}-\textsc{ques}}
{`What did Kim and Lee read?'}
\item[A:]\shortex{3}
{Kim-un & chayk-ul & ilk-ess-e.}
{Kim-\textsc{nun} & book-\textsc{acc} & read-\textsc{pst}-\textsc{decl}} 
{`Kim read a book.' [kor]}}}


\cite{lambrecht:96} regards `\isi{contrastiveness}' as a merely cognitive
concept, yet there are quite a few counterexamples to the claim from a
cross-linguistic perspective. Some languages have a linguistic means
of marking contrast in a distinctive way from non-contrastive \isi{topic}
and \isi{focus}. For instance, \ili{Vietnamese} uses a contrastive-topic
marker \textit{th{\`i}} \citep{nguyen:06} as shown in (\ref{exe:vie}).
The contrast function is shown by the alternative set evoked in
(\ref{exe:vie}), while the distinctiveness from focus is shown by the
fact that \textit{th{\`i}}-marked NPs cannot be used to answer
\textit{wh}-questions.\is{contrast}



\myexe{\enumsentence{\toplabel{exe:vie}
\shortex{4}
  {Nam & th{\`i} & {\textcrd}i & H\`{a} N\textsubdot{\^{o}}i}
  {Nam & \textsc{top} & go & Ha Noi}
  {`Nam goes to Hanoi(, but nobody else).' [vie] \citep[1]{nguyen:06}}}}


\noindent We also find syntactic marking of contrast in several
languages. In \ili{Standard Arabic}, for instance, contrastively
focused items are normally preposed 
to the sentence-initial position,\is{non-contrastive focus}
while non-contrastively focused items which convey new information
(i.e.\ \isi{semantic focus} in \citeauthor{gundel:99}'s terminology) are
in situ with a specific pitch accent, as exemplified in
(\ref{exe:arabic}a--b) respectively \citep{ouhalla:99}.


\myexe{\eenumsentence{\toplabel{exe:arabic}
\item \shortex{3}
  {RIWAAYAT-AN & {\textglotstop}allat-at & Zaynab-u}
  {novel-\textsc{acc} & wrote-she & Zaynab-\textsc{nom}}
  {`It was a \textsc{novel} that Zaynab wrote.'}
\item \shortex{3}
  {{\textglotstop}allat-at & Zaynab-u & RIWAAYAT-an}
  {wrote-she & Zaynab-\textsc{nom} & novel-\textsc{acc}}
  {`Zaynab wrote a \textsc{novel}.' [arb] \citep[337]{ouhalla:99}}}}


\noindent Similarly, in \ili{Portuguese}, contrastive focus precedes
the verb, while non-con\-tras\-tive \isi{focus} follows the verb
\citep{ambar:99}, as exemplified in (\ref{exe:ambar:contrast}A).  If
\textit{a tarte} `a pie' conveys a contrastive meaning as implied in
the translation `What else she ate, I don't know.', it cannot be
preceded by the verb \textit{comeu} `ate'.


\myexe{\eenumsentence{\label{exe:ambar:contrast}
\item[Q:]\shortex{4}
  {Que & comeu & a & Maria?}
  {what & ate & the & Mary}
  {`What did Mary eat?'}
\item[A:]\shortex{5}
  {\#A & Maria & comeu & a & tarte.}
  {the & Mary & ate & the & pie.}
  {`Mary ate the pie (What else she ate, I don't know.)' [por] \citep[28--29]{ambar:99}}}}


\noindent In \ili{Russian}, \isi{contrastive focus} is preposed, while
non-contrastive focus shows up clause-finally
\citep{neeleman:titov:09}.\is{clause-final} For example, in
\myref{exe:neeleman:titov:contrast}, \textsc{jazz-pianista}
`jazz-pianist' in initial position shows a contrast with
\textit{jazz-gitarista} `jazz-guitarist'.\is{contrast}


\myexe{\enumsentence{\toplabel{exe:neeleman:titov:contrast}
\shortexnt{4}
  {\textsc{jazz-pianista} & mal'\v{c}iki & sly\v{s}ali & vystuplenie}
  {jazz-pianist.\textsc{gen} & boys & listened & performance.\textsc{acc}}
\newline
\shortex{3}
  {(a & ne & jazz-gitarista).}
  {(and & not & jazz-guitarist.\textsc{gen})}
  {`The boys listened to the performance of the jazz pianist.' [rus] \citep[519]{neeleman:titov:09}}}}



\subsection{Subtypes of contrast}
\label{3:ssec:subtypes-contrast}

Contrast can be used with either \isi{focus} or \isi{topic}, resulting in two
subtypes: \isi{contrastive focus}  and \isi{contrastive topic}.\is{contrast}  These may 
co-occur in a single clause. For example, in (\ref{exe:contrasts})
taken from \citet{van:05}, Mary and Sally are
contrastively focused, whereas book and magazine are
contrastively topicalized.

\myexe{\eenumsentence{\toplabel{exe:contrasts} \item[Q:] Who did Bill
    give the book to and who did he give the magazine to?  \item[A:]
    He gave the \textbf{book} to \textsc{Mary} and the \textbf{magazine} to
    \textsc{Sally}. \citep[72]{van:05}}}



\subsection{Linguistic properties of contrast}
\label{3:ssec:properties-contrast}


In addition to the distributional facts presented in
\S\ref{3:ssec:definition-contrast}, which substantiate the existence
of contrast as a component of information structure,\is{contrast} there is also an
argument that contrast behaves differently from \isi{non-contrastive focus}
(or topic) in the semantics.\is{non-contrastive topic}


\cite{gundel:99} provides several differences between contrastive
\isi{focus} and non-contrastive focus, as already presented in
Section \ref{3:sssec:gundel}.  The different behavior between them is also
exemplified in \myref{exe:partee:prague} taken from
\citet{partee:91}. (\ref{exe:partee:prague}a) can be ambiguously
interpreted depending on where the accent is
assigned. (\ref{exe:partee:prague}d), in which the subscript
\mysub{\textsc{cf}} stands for a specific accent responsible for
contrastive focus, has the same truth-conditions as
(\ref{exe:partee:prague}c), but not
(\ref{exe:partee:prague}b).\is{truth-conditions}



\myexe{\eenumsentence{\toplabel{exe:partee:prague}
\item The largest demonstrations took place in Prague in November (in) 1989.
\item The largest demonstrations took place in \textsc{Prague} in November (in) 1989.
\item The largest demonstrations took place in Prague in \textsc{November} (in) 1989.
\item The largest demonstrations took place in \textsc{Prague}\mysub{\textsc{cf}} in \textsc{November} (in) 1989. \citep[301--302]{gundel:99}}}



\citet{nakanishi:07} compares contrastive \isi{topic} with thematic topic
(also known as non-contrastive topic or aboutness topic in the present
study) in \ili{Japanese} from several angles,\is{aboutness topic} since \wa can be used
for either the theme or the contrastive element of the sentence. From
a distributional viewpoint, a non-contrastively \wa-marked
constituent can be either anaphoric (i.e.\ previously mentioned) or
generic, whereas a contrastive element with \wa can be generic,
anaphoric, or neither. \myref{exe:nakanishi:dist} is an example in
which a contrastively \wa-marked NP conveys neither an anaphoric
interpretation nor a generic one.\is{contrast}


\myexe{\enumsentence{\label{exe:nakanishi:dist}
\evnup{\begin{tabular}[h]{llllllllll}
oozei-no & hito-wa & paatii-ni & kimasi-ta \\
many-\textsc{gen} & people-\textsc{wa} & party-to & come-\textsc{pst}\\
\end{tabular}}
\newline
\evnup{\begin{tabular}[h]{llllllllll}
ga & omosiroi & hito-wa & hitori & mo & imas-en-desita.\\
but & interesting & people-\textsc{wa} & one-person & even & be-\textsc{neg}-\textsc{pst}\\
\multicolumn{10}{l}{`Many people came to the party indeed but there was none who was} \\
\multicolumn{10}{l}{interesting.' [jpn] \citep[270]{kuno:73}} \\
\end{tabular}}}}


\noindent From a phonological stance, if \wa is used for a thematic
interpretation, the highest value of F0 contour after \wa is as high
as or even higher than the highest value before \wa.\il{Japanese}
In contrast, if it denotes a contrastive meaning,\is{contrast}
\wa is realized with a dramatic
downslope of F0 contour.  From a semantic perspective, it turns out
that the two versions of the marker have different scopal
interpretations when they co-occur with \isi{negation}. The scopal
interpretation driven by the relationship with \isi{focus} and negation was
originally captured by \citet{buring:97} as exemplified earlier in
\myref{exe:buring:scope}. \citeauthor{nakanishi:07}, in line with the
claim of \citeauthor{buring:97}, compares two types of
\textit{wa}-marked topics in \ili{Japanese} as shown in
\myref{exe:nakanishi:scope}: Thematic \wa in
(\ref{exe:nakanishi:scope}a) and contrastive \wa in
(\ref{exe:nakanishi:scope}b) have the opposite scopal reading to each
other.





\myexe{\eenumsentence{\label{exe:nakanishi:scope}
\item\shortex{2}
  {Minna-wa & ne-nakat-ta.}
  {everyone-\textsc{wa} & sleep-\textsc{neg}-\textsc{past}}
  {`Everyone didn't sleep.' \\
    (thematic \wa) \ensuremath{\surd}\ensuremath{\forall}\ensuremath{>}\ensuremath{\neg}, *\ensuremath{\neg}\ensuremath{>}\ensuremath{\forall}}
\item\shortex{2}
  {[Minna-wa]\mysub{T} & ne-[nakat]\mysub{F}-ta.}
  {everyone-\textsc{wa} & sleep-\textsc{neg}-\textsc{past}}
  {`Everyone didn't sleep.' \\
    (contrastive \wa) *\ensuremath{\forall}\ensuremath{>}\ensuremath{\neg}, \ensuremath{\surd}\ensuremath{\neg}\ensuremath{>}\ensuremath{\forall} \citep[187--188]{nakanishi:07}}}}




Compared to non-contrastive topics, contrastive topics tend to have
relatively weak constraints on positioning and the selection of
NP. \citet{choi:99} provides an analysis of \isi{scrambling}
(i.e.\ OSV) in \ili{Korean}, which reveals that contrastive \isi{focus} can
freely scramble, while completive focus (also known as \isi{non-contrastive focus}
or \isi{semantic focus}) cannot scramble. \citet{erteschik:07} argues that
in \ili{Danish} contrastive topic can be associated with non-specific
indefinites, whereas non-contrastive \isi{topic} cannot, as shown in
\myref{exe:erteschik:8-9}. \textit{En pige} `a girl' in
(\ref{exe:erteschik:8-9}a) cannot play the non-contrastive topic role,
because its interpretation is non-specifically indefinite. In
contrast, \textit{et museum} `a museum' in (\ref{exe:erteschik:8-9}b)
can be the \isi{topic} of the sentence, because it has an alternative
\textit{en kirke} `a church'.

\myexe{\eenumsentence{\label{exe:erteschik:8-9}
\item\shortex{5}
  {\#En & pige & m{\o}dte & jeg & i g\r{a}r.}
  {a & girl & met & I & yesterday}
  {`I met a girl yesterday.'}
\item\shortexnt{6}
  {Et & museum & bes{\o}gte & jeg & allerede &  i g\r{a}r,}
  {a & museum & visited & I & already & yesterday}
\vspace{-10pt}
\item[]\shortex{6}
  {en & kirke & ser & jeg & f{\o}rst & i morgen.} 
  {a & church & see & I & only & tomorrow}
  {`I visited a museum already yesterday, I will see a church only tomorrow.' [dan] \citep[8--9]{erteschik:07}}}}



\subsection{Tests for contrast}
\label{3:ssec:tests-contrast}

\citet[42--43]{gryllia:09} provides six tests to vet the meaning and marking
of \isi{contrastive topic} and \isi{contrastive focus} as follows.\footnote{For 
more elaborated explanation and examples
  for each of them, see Chapter 3 in \citet{gryllia:09}.  This
  subsection, for brevity, provides only the definition and
  representative examples focusing on correction test.\is{correction test}}\is{contrast}



\myexe{\eenumsentence{\toplabel{test:gryllia:contrastive}
\item \textit{Wh}-questions: A contrastive answer is not compatible
  with a common \textit{wh}-question.\is{\textit{wh}-questions}
\item Correction test: A contrastive focus can be used to answer a
  yes-no question correcting part of the predicate information of the
  question.\is{correction test}
\item Choice test: When answering an alternative question, one
  alternate is contrasted to the other.
\item Accommodation focus test: When the discourse is accommodated in
  such a way that the initial \textit{wh}-question can be interpreted
  as containing a positive and a negative question
  (e.g.\ \textit{who came?}, \textit{who did not come?}), then
  the focus in the answer is contrastive.\is{negation}
\item Substitution test: If two terms are interpreted with a `List
  Interpretation', then they can be substituted with \textit{the former}
  and \textit{the latter}.
\item Right dislocation: Contrast is incompatible with right
  dislocation.\is{right dislocation test}
\item Implicit subquestion test: (i) When a \textit{wh}-question can
  be split into subquestions and the answer is organized per
  subquestion, then, there is a contrastive topic in the answer.
  (ii) When a question can be interpreted as containing more than
  one implicit subquestion, and the answer addresses only one of
  these subquestions, rather than the general question, then, this
  answer contains a contrastive topic.}}



Some of the diagnostics above, however, are not cross-linguistically
valid. In non-Indo-European languages, such as \ili{Korean}, only some
work. For example, the \textit{wh}-question test does not work for
\ili{English} and Korean in the same manner, as exemplified below.\is{\textit{wh}-test}


\myexe{\eenumsentence{\toplabel{exe:eng:contrastive}
\item[Q:] Who came?
\item[A:] Well, \textsc{Kim} came, I know that much, but I can't tell you about anyone else.}}

\myexe{\eenumsentence{\toplabel{exe:kor:contrastive}
\item[Q:]\shortex{2}
{nwuka & o-ass-ni?}
{who & come-\textsc{pst}-\textsc{ques}}
{`Who came?'}
\item[A:]\shortex{2}
{Kim-i/un & o-ass-e.}
{Kim-\textsc{nom}/\textsc{nun} & come-\textsc{pst}-\textsc{decl}}
{`Kim came.' [kor]}}}


\noindent \textit{Kim} with \nun in (\ref{exe:kor:contrastive}A) can
be an appropriate answer to the question, and it involves a
contrastive interpretation (i.e.\ conveying a meaning like \textit{I
  know that at least Kim came, but I'm not sure whether or not others
  came.}).\is{contrast}  In this case, the replier alters the information
structure articulated by the questioner arbitrarily in order to offer
a more informative answer to the solicited question. Note that
\isi{contrastiveness} is basically speaker-oriented \citep{chang:02}. In
other words, contrast is primarily motivated by the speaker's
necessity to attract the hearer's special attention at a particular
point in the discourse. Thus, speakers may change the stream of
information structure as they want.
	
The \isi{right dislocation test}, on the other hand, seems valid in
\ili{Korean} as well. The \onun-marked NP can be used in the right
dislocation constructions in Korean as given in
(\ref{exe:kor:right-dislocation}Q1). Yet, if an \isi{alternative set} is
entailed as shown in (\ref{exe:kor:right-dislocation}Q2), 
\isi{right dislocation} sounds absurd.

\myexe{\eenumsentence{\toplabel{exe:kor:right-dislocation}
\item[Q1:]\shortex{2}
{cangmi-nun & ettay?}
{rose-\textsc{nun} & about}
{`How about the rose?'}
\item[A:]\shortex{2}
{coh-a, & cangmi-nun.}
{good-\textsc{decl} & rose-\textsc{nun}}
{`It's good, the rose.'}
\item[Q2:]\shortex{2}
{kkoch-un & ettay?}
{flower-\textsc{nun} & about}
{`How about the flowers?'}
\item[A1:]\shortexnt{2}
{\#coh-a, & cangmi-nun.}
{good-\textsc{decl} & rose-\textsc{nun}}
\item[A2:]\shortexnt{2}
{cangmi-nun & coh-a.}
{rose-\textsc{nun} & good-\textsc{decl} [kor]}}}


The most convincing and cross-linguistically applicable tests among
the tests in \myref{test:gryllia:contrastive} is the correction test,
as exemplified in \myref{exe:correction:1} in \ili{Italian}
\myref{exe:correction:2} in \ili{Greek}, and \myref{exe:correction:3}
in \ili{Korean}.


\myexe{\eenumsentence{\toplabel{exe:correction:1}
\item[Q:]\shortex{6} {L' & ha & rotto & Giorgio, & il & vaso?}  {it &
  has & broken & Giorgio & the & vase} {`Has Giorgio broken the
  vase?'}
\item[A:]\shortex{5} {[Maria]\mysub{C-Foc} & ha & rotto & il
  & vaso.}  {Maria & has & broken & the & vase} {`It is Maria who has
  broken the vase.' [ita] \citep[32]{gryllia:09}}}}

\myexe{\eenumsentence{\toplabel{exe:correction:2}
\item[Q:]\shortex{2}
{Thelis & tsai?}
{want.2\textsc{sg} &  tea.\textsc{acc}}
{`Would you like tea?'}
\item[A1:]\shortex{3}
{Ohi, & thelo & [kafe]\mysub{C-Foc}.}
{no & want.1\textsc{sg} & coffee.\textsc{acc}}
{`No, I would like coffee.'}
\item[A2:]\shortexnt{3}
{Ohi, & [kafe]\mysub{C-Foc} & thelo}
{no & coffee.\textsc{acc} & want.1\textsc{sg} [ell] \citep[44]{gryllia:09}}}}

\myexe{\eenumsentence{\toplabel{exe:correction:3}
\item[Q:]\shortex{2}
{chayk & ilk-ess-ni?}
{book &  read-\textsc{pst}-\textsc{ques}}
{`Did you read a book?'}
\item[A:]\shortex{3}
{ani, & capci-lul/nun & ilk-ess-e.}
{no & magazine-\textsc{acc}/\textsc{nun} & read-\textsc{pst}-\textsc{decl}}
{`No, (but) I read a magazine.' [kor]}}}


\citet{gussenhoven:07}, in a similar vein, suggests corrective \isi{focus}
as a subtype of focus in \ili{English} as presented below.


\myexe{\eenumsentence{\toplabel{exe:gussenhoven:corrective}
\item[A:]{What's the capital of Finland?}
\item[B:]{The CAPital of FINland is [HELsinki]\mysub{FOC}}
\item[A$^\ensuremath{\prime}$:]{The capital of Finland is OSlo.}
\item[B$^\ensuremath{\prime}$:]{(NO.) The capital of Finland is [HELsinki]\mysub{CORRECTIVE}
  \citep[91]{gussenhoven:07}}}}


\noindent \citeauthor{gussenhoven:07} also provides a similar example
in \ili{Navajo}.\is{negation} Navajo has two negative modifiers; one is neutral,
\textit{doo} ... \textit{da} in (\ref{exe:nav}a), and the other expresses
corrective \isi{focus}, \textit{hanii} in (\ref{exe:nav}b). That is,
\textit{hanii} serves to mark a \isi{contrastive focus} in \ili{Navajo}.

\myexe{\eenumsentence{\label{exe:nav}
\item\shortex{4}
{J\'aan & doo & chid\'i & yiy\'i\'i{\l}ch{\o}'-da.}
{John & \textsc{neg} & car & 3\textsc{rd}.\textsc{past}.wreck-\textsc{neg}}
{`John didn't wreck the car.'}
\item\shortex{4}
{J\'aan & hanii & chid\'i & yiy\'i\'i\l ch{\o}'.}
{John & \textsc{neg} & car & 3\textsc{rd}.\textsc{past}.wreck}
{`\textsc{John} didn't wreck the car (someone else did).' [nav] \citep[91]{gussenhoven:07}}}}



\citet{wee:01} proposes a test with conditionals, in which a contrastive \isi{topic}
is paraphrased into a conditional clause as exemplified in
(\ref{exe:wee}B$^\ensuremath{\prime}${}$^\ensuremath{\prime}$).\is{correction
  test} That is, \nun which can convey contrast meaning in
\ili{Korean} can be altered into a conditional marker \textit{lamyen},
which also has an alternative set drawn by \textit{nobody} in
(\ref{exe:wee}A) and functions to make a correction to the
presupposition given in (\ref{exe:wee}A).


\myexe{\eenumsentence{\label{exe:wee}
\item[A:]{Nobody can solve the problem.} 
\item[B:]{Peter would solve the problem.}
\item[B$^\ensuremath{\prime}$:]\shortex{4}
{Peter-nun & ku & muncey-lul & phwul-keya.}
{Peter-\textsc{nun} & the & problem-\textsc{acc} & solve-would}
{`Peter would solve the problem.'}
\item[B$^\ensuremath{\prime}${}$^\ensuremath{\prime}$:]\shortex{4}
{Peter-lamyen, & ku & muncey-lul & phwul-keya.}
{Peter-if & the & problem-\textsc{acc} & solve-would}
{`If Peter were here, he would solve the problem.' [kor]}}}


\noindent \citet{fintel:04} and \citet{kim:12b} suggest a test for
contrast called ``Hey, wait a minute!'', which serves to cancel or
negate presupposed content in the previous discourse. In other words,
the contrastive marking acts as the key for correcting the inaccurate
part in a presupposition. Likewise, \citet{skopeteas:fanselow:10},
exploring focus positions in Georgian, define contrastive \isi{focus} as a
``corrective answer to truth value question''.  This definition is
also in line with my argument that the correction test can be reliably
used to vet \isi{contrastive focus}.













The present study makes use of the correction test to scrutinize
contrast.  However, that does not means that recognizing corrections
is the only use of \isi{contrastive focus}. Note that use for corrections is
a sufficient condition for expressing contrastive focus, but not a
necessary condition.


Lastly, because foci are inomissible while topics are not, if a
constituent that passes the correction test cannot be
elided,\is{correction test} it is evaluated as conveying contrastive
\isi{focus}.  If a constituent passes the correction test but can be
dropped, it is regarded as contrastive \isi{topic}.








\section{Background}
\label{3:sec:background}


We can say a constituent is in the \isi{background} when it conveys a
meaning of neither \isi{focus} nor \isi{topic}.  In terms of linguistic forms,
background constituents typically do not involve additional marking
but may be forced into particular positions in a sentence. Background
is in complementary distribution to both topic and focus and adds no
information structure meaning to the discourse. Focus, topic, and
background are mutually exclusive, and thereby cannot overlap with
each other.\footnote{As aforementioned, there exists an opposing view
  to this generalization \citep{krifka:08}.}



Background can often be found in cleft
sentences.\is{background}\is{clefting} Clefts refer to (\isi{copula})
constructions consisting of a main clause and a dependent clause
(e.g.\ a \isi{relative clause}),\is{narrow focus} in which a constituent in the main
clause is narrow-focused.\footnote{The focused item in clefts does not
  need to be an argument \isi{focus}, because non-nominal categories such as
  adverbs and information status is represented can sometimes take place in the main clause of
  clefts as given in \myref{exe:clefts:narrow2}.}  The narrow foci in
cleft constructions can be easily identified by means of a deletion
test.\is{deletion test} As noted before, focus means a constituent
that can never be elided, which is one of the main behaviors
distinguishing focus from topic and background. Thus, any other
constituent in (\ref{exe:clefts:narrow1}-\ref{exe:clefts:narrow2}),
except for the narrowly focused ones \textit{Kim} and \textit{from
  her}, can be freely eliminated.

\myexe{\enumsentence{\toplabel{exe:clefts:narrow1}
\begin{tabular}[t]{ll}
Q: & {Who reads the book?}\\
A1: & {It is Kim that reads the book.}\\
A2: & {It is Kim.}\\
A3: & {Kim.}\\
\end{tabular}}}

\myexe{\enumsentence{\toplabel{exe:clefts:narrow2}
\begin{tabular}[t]{ll}
Q: & {Where did you have my address from?}\\
A1: & {It was from her that I had your address.}\\
A2: & {It is from her.}\\
A3: & {From her.}\\
\end{tabular}}}



Clefts typically put the part of the sentence after the focused item
into \isi{background}.\is{clefting} Since the remaining part of the
sentence (i.e.\ the cleft clause) such as \textit{that reads the
  book} and \textit{that I had your address} in each cleft sentences
can be freely dropped, it can be regarded as either \isi{topic} or
background.  Moreover, the constituents in cleft clauses are rarely
\onun-marked in \ili{Korean}, as shown in \myref{exe:clefts:kor}.




\myexe{\eenumsentence{\toplabel{exe:clefts:kor}
\item\shortex{5}
  {ku & chayk-ul/*un & ilk-nun & salam-i/un & Kim-i-ta.}
  {the & book-\textsc{acc}/\textsc{nun} & read-\textsc{rel} & person-\textsc{nom}/\textsc{nun}  & Kim-\textsc{cop}-\textsc{decl}}
  {`It is Kim that reads the book.'}
\item\shortex{5}
  {Kim-i/*un & ilk-nun & kes-i/un & ku & chayk-i-ta.}
  {Kim-\textsc{nom}/\textsc{nun} & read-\textsc{rel} & thing-\textsc{nom}/\textsc{nun} & the & book-\textsc{cop}-\textsc{decl}}
  {`It is the book that Kim reads.' [kor]}}}


\noindent As discussed thus far, \nun in \ili{Korean} assigns either
topic or contrast, or both (i.e.\ contrastive topic) to the word it is
attached to.\is{clefting}\is{contrast} The marker \nun cannot be used within the
cleft clauses as shown in \myref{exe:clefts:kor}. Thus, NPs in cleft
clauses are usually identified as \isi{background} (i.e.\ non-focus
and non-topic, simultaneously), at least in Korean.  Cleft clauses can
contain a focused constituent in some languages as exemplified in
\myref{exe:gussenhoven:clefts}, however. 


\myexe{\eenumsentence{\label{exe:gussenhoven:clefts} 
\item[Q:]{Does Helen know \textsc{John}?} 
\item[A:]{It is John/\textsc{John} she \textsc{dislikes}.}
\item[Q:]{I wonder who she dislikes.} 
\item[A:]{It is \textsc{John} she dislikes. \citep[96]{gussenhoven:07}}}}


\noindent Thus, we cannot say cleft clauses are always in background,
and more discussion about cleft clauses is given in
Section \ref{10:sssec:clefts:cleft} \mypage{10:sssec:clefts:cleft}.


\section{Summary}
\label{3:sec:summary}


This chapter has reviewed the primary components of information structure
(\isi{focus}, \isi{topic}, contrast and \isi{background}), including definitions
of the concepts and explorations of sub-classifications, associated
linguistic phenomena and potential tests.  The assumptions presented
in the previous section are as follows: First, I establish that
information status is not a reliable means of identifying information
structure since the relationship between the two is simply a tendency.
Second, I define focus as what is new and/or important in a sentence,
and specify that a constituent associated with focus cannot be
eliminated from a sentence. I present two subtypes of focus; \isi{semantic focus}
(lacking a contrastive meaning) and \isi{contrastive focus}.\is{\textit{wh}-questions}
Tests to vet focus marking and
meaning include \textit{wh}-questions and the deletion
test.\is{deletion test} Third, I define topic as what a speaker is
talking about. While every sentence presumably has at least one focus,
topic may or may not appear in the surface form. I outline two
subtypes; aboutness topic (also known as thematic topic or non-contrastive
topic) and contrastive topic. Frame-setters, which serve to restrict
the domain of what is spoken (temporal, spatial, conditional, manner,
etc.), are always external (not an argument of the predicate) and
sentence-initial.  In contrast to previous work, frame-setters are not
treated as a subtype of topic here.  Because the semantic core of
topic is aboutness,\is{aboutness} the tools for identifying topics are the
\textit{tell-me-about} test and several paraphrasing tests such as
\textit{as for ...}, \textit{speaking of ...}, and \textit{(what)
  about ...}.\is{\textit{tell-me-about} test}\is{contrast} Next, I explicate the
ways in which contrast always entails an \isi{alternative set}, which
can be realized as either \isi{contrastive focus} or \isi{contrastive topic}. The
most reliable and cross-linguistically valid diagnosis for contrast is
the correction test,\is{correction test} because correction
necessarily requires an alternative. Finally, I define background as
neither focus nor \isi{topic}, and posit that any constituent associated
with it can be freely elided without loss of information
delivery. These cross-linguistic generalizations help provide
linguistic generalizations to be used in creating HPSG/MRS-based
constraints on information structure.\is{HPSG}\is{MRS} Moreover, they
are also used to design the library of information structure for the
\lingo \isi{Grammar Matrix} system.\is{frame-setter}




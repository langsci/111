\chapter{Introduction}
\label{chapter1}
\setcounter{enums}{0}

Human languages consist of various structures, among which syntactic
structure and semantic structure are particularly well-known. The
present study is primarily concerned with information structure, and
the ways in which it could be leveraged in natural language processing
applications.

Information structure is realized by prosodic, lexical, or syntactic
cues which constrain interpretation to meet communicative demands
within a specific context.  Information structure is comprised of four
primary components: \isi{focus}, \isi{topic}, \isi{contrast}, and
\isi{background}.  Focus marks what is new and/or important in a
sentence, while topic marks what the speaker wants to talk about.
Contrast, realized as either \isi{contrastive focus} or contrastive topic,
signals a contextually contrastive set of \isi{focus} or topic items
respectively.\is{contrast} That which is not marked as either focus or topic is
designated as background information.

Information structure affects the \isi{felicity} of using a sentence in
different discourse contexts, as exemplified in \myref{exe:intro}.

\myexe{\eenumsentence{\toplabel{exe:intro}
\item{Kim reads the book.}
\item{It is Kim that reads the book.}
\item{It is the book that Kim reads.}}}



\noindent Though the sentences in (\ref{exe:intro}b-c) are constructed
using the same lexical items and describe the same state of affairs as
sentence (\ref{exe:intro}a), they differ with respect to how
information is packaged: `Kim' is focused in (\ref{exe:intro}b), while
`the book' is focused in (\ref{exe:intro}c). This difference in
information structure means that (\ref{exe:intro}b) would be a
felicitous answer to \textit{Who is reading the book?} and
(\ref{exe:intro}c) would be a felicitous answer to \textit{What is
  that book?}, but not vice versa.


Furthermore, information structure can be key to finding felicitous
translations \citep{paggio:96,kuhn:96}.  Since languages vary in the
ways they mark information structure, a model of information structure
meanings and markings is a key component of a well-constructed
grammar.  For example, the simple \ili{English} sentence
(\ref{exe:i-am-kim}a) can be translated into at least two
\ili{Japanese} allosentences (close paraphrases which share
\isi{truth-conditions} \citep{lambrecht:96}), with the nominative
marker \ga or with the so-called \isi{topic} (and/or contrast) marker \wa.\is{contrast}


\myexe{\eenumsentence{\label{exe:i-am-kim}
\item I am Kim.
\item\shortexnt{4}
{watashi & ga/wa & Kim & desu.}
{I & \textsc{nom}/\textsc{wa} & Kim & \textsc{cop} [jpn]}}}


\noindent The choice between \isi{alternatives} is conditioned by
context. Marking on the NP hinges on whether \textit{watashi} `I'
functions as the \isi{topic} or not. If the sentence is an answer to a
question like \textit{Who are you?}, \wa is preferred. If the sentence
is instead a reply to a question like \textit{Who is Kim?}, answering
using \textit{wa} sounds unnatural.


This difference in \isi{felicity-conditions} across languages should
be taken into consideration in computational linguistics; especially
in machine translation.  When machine translation systems cannot
accurately model information structure, resulting translations may
sound rather unnatural to native speakers.  Successful translation
requires reshaping how information is conveyed in accordance with the
precepts of the target language and not simply changing words and
reordering phrases in accordance with syntactic rules. Better accuracy
in translation requires the incorporation of information structure.




\section{Motivations}
\label{1:sec:motivations}


The nature of information structure is less understood than that of
syntactic and semantic structures. For many languages the full range
of information structure markings remains unknown. Furthermore, the
integration of information structure has been rather understudied in
computational linguistic investigations to date, despite its potential
for improving machine translation, Text-to-Speech, automatic text
summarization, and many other language processing systems.



There are several opportunities for improved incorporation of further
exploration of information structure.  First, the absence of
cross-linguistic findings results in less use of the information in
practical systems.  In order for language processing systems to
provide more fine-grained results and work truly
language-independently, acquiring knowledge across languages are
required \citep{bender:11}.  Second, distributional findings obtained
from language data are still insufficient. Several previous studies
exploit language data for the study of information structure, but use
merely monolingual or bilingual texts
\citep{komagata:99,johansson:01,bouma:etal:10,hasegawa:koenig:11}, so
a larger picture of how information structure works across a range of
languages is still elusive.  Third, existing proposals for
representing information structure within a grammatical framework for
natural language processing remain somewhat underdeveloped and
insufficiently tested.  Previous literature, including
\citet{king:97}, \citet{steedman:00}, and \citet{bildhauer:07},
provides a variety of formalisms to represent information structure,
but none of them has been shown to be cross-linguistically
valid. Moreover, their formalisms have never been implemented into a
practical system in a fully comprehensive way.  Lastly, largely for
the reasons presented thus far, the potential improvement to machine
translation and language processing systems derivable from using
information structure has not yet been shown. If the contribution of
information structure to improvement of practical applications were to
be quantitatively substantiated through experimentation, this would
motivate further development of information structure-based
applications.


The central goal of this book is to create a computational model to
handle information structure from a multilingual perspective within
the \isi{HPSG} (Head-driven Phrase Structure Grammar,
\citealt{pollard:sag:94}) framework, using the \isi{MRS} (Minimal
Recursion Semantics, \citealt{copestake:etal:05}) formalism, and
contributing to the \lingo \isi{Grammar Matrix} system
(\citealt{bender:etal:10}).



\section{Grammar Engineering}
\label{2:sec:grammar-engineering}



In a nutshell, \isi{grammar engineering} is the process of creating
machine-readable implementations of formal grammars. In order to
understand what grammar engineering is, it is necessary to define what
language is.  Since the early days of generative study in linguistics,
language has been defined as (i) an infinite set of (ii) strings
accepted as (iii) grammatical by (iv) native speakers, and grammar
engineering has embraced this definition.  (i) Given that the number
of sentences in human language is assumed to be nonfinite, grammar
engineering takes the generative capacity of grammar into account in
sentence-generation as well as sentence-parsing. (ii) Most work in
grammar engineering is based on text-processing in that language is
regarded as a set of strings. (iii) Since formulating grammatical
well-formedness in a language is crucial, grammar engineering is
fundamentally concerned with constructing a linguistically-precise and
broad-coverage grammar. (iv) Finally, grammaticality has to be judged
by native speakers. The judgment can be made either via linguistic
intuition or with reference to language data such as corpora.
Intuition-based and data-based methodologies complement each other in
grammar engineering.\footnote{\citet{balwin:etal:05}, in the context
  of grammar engineering, discuss this spirit in an overall sense and
  conduct an evaluation using the ERG (English Resource Grammar,
  \citealt{flickinger:00}) and the BNC (British National Corpus,
  \citealt{burnard:00}). They substantiate the interaction of two
  sources of linguistic findings: namely acceptability judgments and
  corpus data.}




The main goal of grammar engineering is to build up reusable
computational grammar resources.\is{grammar engineering} Ideally, the
empirical description of the grammar resources is linguistically
motivated. A grammar is to be described in a linguistically
well-elaborated way, and on a large enough scale to cover the
linguistic phenomena in a human language.  For this purpose, grammar
engineering also utilizes various types of linguistic data such as
(machine-readable) dictionaries, corpora
\citep{nichols:etal:10,song:etal:10}, testsuites \citep{oepen:01},
treebanks \citep{oepen:etal:04,bond:etal:06}, and wordnets
\citep{bond:etal:09,pozen:13}.  The described grammar should be able
to run on a computer in order to prove its mathematical tractability
as well as its potential for utilization. The constructed grammar has
to be reusable for other studies with varied research
goals.\footnote{\citet{bender:08} offers an explanation about how
  grammar engineering can be used for linguistic hypothesis testing:
  ``[L]anguages are made up of many subsystems with complex
  interactions. Linguists generally \isi{focus} on just one subsystem at a
  time, yet the predictions of any particular analysis cannot be
  calculated independently of the interacting subsystems. With
  implemented grammars, the computer can track the effects of all
  aspects of the implementation while the linguist focuses on
  developing just one.'' (Ibid.\ p.\ 16)} Buidling upon the grammar
resources, grammar engineering facilitates parsing and generation,
which can be used for several practical applications such as machine
translation, grammar checking, information extraction,
question-answering, etc.




Within the field of grammar engineering, there are several competing
theories of grammar, including \isi{HPSG}, \isi{LFG}
(Lexical-Functional Grammar (\citealt{bresnan:01})), \isi{CCG}
(Combinatory Categorial Grammar, \citealt{steedman:01}), and \isi{TAG}
(Tree-Adjoining Grammar, \citealt{joshi:schabes:97}). HPSG, which
employs typed feature structures as a mathematical foundation, has
been used for creation of reusable computational grammars in many
languages. Those who study grammar engineering within the HPSG
framework have cooperated with each other, by forming a consortium
called \isi{DELPH-IN} (DEep Linguistic Processing with HPSG -
INitiative, \myurl{http://www.delph-in.net}).\footnote{There are other
  initiatives based on HPSG as well as other frameworks, such as
  CoreGram for HPSG-based implementations
  (\url{http://hpsg.fu-berlin.de/Projects/CoreGram.html}) using the
  TRALE system
  (\url{http://www.sfs.uni-tuebingen.de/hpsg/archive/projects/trale}),
  and ParGram in LFG-based formalism
  (\url{http://pargram.b.uib.no}). There are also other HPSG-based
  grammars such as Enju for English
  (\url{http://www.nactem.ac.uk/enju}, \citealt{miyao:tsujii:08}), and
  a Chinese grammar constructed in a similar way to Enju
  (\citealt{yu:etal:10}).}  DELPH-IN, in the spirit of open-source NLP
\citep{pedersen:08}, provides research and development outcomes in a
readily available way. These are largely gathered in the \logon
repository (\myurl{http://moin.delph-in.net/LogonTop}).\footnote{Note
  that not all DELPH-IN resources are in the \logon repository. For
  example, the collection of \textit{Language CoLLAGE} is not in the
  repository, but is readily available \citep{bender:14}.} \logon
includes a collection of computational grammars (e.g.\ \isi{ERG} for
English (\citealt{flickinger:00}), \isi{Jacy} for \ili{Japanese}
(\citealt{siegel:bender:02}), \isi{KRG} for \ili{Korean}
(\citealt{kim:etal:11}), \isi{GG} for \ili{German}
(\citealt{crysmann:03,crysmann:05a,crysmann:05b}), \isi{SRG} for
\ili{Spanish} (\citealt{marimon:12}), \isi{LXGram} for
\ili{Portuguese} (\citealt{branco:costa:10}), \isi{Norsource} for
\ili{Norwegian} (\citealt{hellan:05}), \isi{BURGER} for
\ili{Bulgarian} (\citealt{osenova:11}), ZHONG \ensuremath{[\big|]} for
the \ili{Chinese} languages (\citealt{fan:15a,fan:15b}),\is{ZHONG}
\isi{INDRA} for \ili{Indonesian} (\citealt{moeljadi:15}), and so
forth), processors (e.g.\ \isi{\lkb}~(\citealt{copestake:02}),
\isi{\pet}~(\citealt{callmeier:00}), etc.), and other software
packages (e.g.\ \isi{\itsdb} (\citealt{oepen:01})).





One of the major projects under the DELPH-IN consortium is the \lingo
Grammar Matrix (\citealt{bender:etal:10}).  The \lingo Grammar Matrix
customization system is an open source starter kit for the rapid
development of HPSG/MRS-based grammars
(\myurl{http://www.delph-in.net/matrix/customize)}).  The grammars
created by the system are to be rule-based, scalable to
broad-coverage, and cross-linguistically comparable. The main idea
behind the system is that the common architecture simplifies exchange
of analyses among groups of developers, and a common semantic
representation speeds up implementation of multilingual processing
systems such as machine translation.



The current work is largely dependent upon the results of the
\isi{DELPH-IN} consortium.  First, I make use of the DELPH-IN
formalism to construct the HPSG/MRS-based information structure
library from a multilingual perspective on grammar
engineering. Second, I refer to the comprehensive DELPH-IN grammars
(i.e.\ resource grammars, such as the \isi{ERG} and the \isi{Jacy})
during the construction. Finally, I utilize the DELPH-IN tools to
check the feasibility of what I propose and conduct several types of
evaluations.





\section{Outline}
\label{1:sec:outlines}

This book is divided into three parts, dedicated to exploring
solutions to each of the problems mentioned in \myS{1:sec:motivations}
individually from a perspective of grammar engineering
(\myS{2:sec:grammar-engineering}).


The first part explores various information structure meanings and
markings and how they are related to each other within and across
different languages.  This is done through a review of previous
studies on information structure as well as through a survey of
various types of languages and their information structure
systems. Building on this initial work and additional evidence, a more
cross-linguistically valid explanation of information structure is
provided. Chapter~\ref{chapter3} lays out the meanings each component
of information structure conveys, and Chapter~\ref{chapter4} looks
into three forms of expressing information structure: namely prosody,
lexical markings, and sentence position.  Chapter~\ref{chapter5}
discusses the discrepancies in meaning-form mapping of information
structure.




The second part presents a formal architecture for representing
information structure within the HPSG/MRS-based formalism.  Several
previous studies on information structure are surveyed in
Chapter~\ref{chapter8}.  After that, I propose the definition of a new
constraint type and feature hierarchy for modeling information
structure in HPSG/MRS. \isi{ICONS} (mnemonic for \isi{Individual
  CONStraints}) is presented as an extension to MRS in
Chapter~\ref{chapter9} and~\ref{chapter10}.  Chapter~\ref{chapter9}
presents the fundamentals of representing information structure via
ICONS, and Chapter~\ref{chapter10} goes into the particulars of how
ICONS works with some sample derivations.  Chapter~\ref{chapter10-2}
shows how information structure in multiclausal utterances can be
represented via ICONS, and Chapter~\ref{chapter10-3} delves into
several means of expressing information structure with reference to
ICONS.  Chapter~\ref{chapter10-4} explores how \isi{focus} projection can be
supported by \isi{underspecification}.\is{focus projection}


The third part is devoted to the implementation of an information
structure-based computational model and evaluates the model.  The
present study is concerned with the \lingo Grammar Matrix system,
especially aiming to create a library for information structure and
add that library into the customization system
(\citealt{bender:flickinger:05,drellishak:09,bender:etal:10}). I
discuss how the library for information structure is built up and how
an information structure-based system works for multilingual machine
translation.  Chapter~\ref{chapter11} builds up a grammar library for
information structure and Chapter~\ref{chapter12} addresses how
machine translation can be improved using ICONS.



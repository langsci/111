\documentclass{article}

\pagestyle{empty}


\usepackage{tikz}
\usetikzlibrary{shapes,backgrounds}
\begin{document}
\pagestyle{empty}
% Suppose we have three circles or ellipses or whatever. Let us define
% commands for their paths since we will need them repeatedly in the
% following:


\def \firstcircle{ (0,0) circle (1cm) }
\def \secondcircle{ (1.5,0) circle (1cm) }
\def \thirdcircle{ (60:1.5) circle (1cm) }
\def \top{ (-2, -1.5) rectangle (3, 2.5) }


% Now we can draw the sets:
\begin{tikzpicture}
    \draw \top node[below left]{{\small plain}};
    \draw \firstcircle (0,0) circle (1cm) node[below] {{\small fronting \mbox{}}};
    \draw \secondcircle (1.5,0)  circle (1cm) node[below] {{\small cleft}};
    \draw \thirdcircle (60:1.5) circle (1cm) node[above] {{\small passive}};

    % Now we want to highlight the intersection of the first and the
    % second circle:

    \begin{scope}
      \clip \firstcircle;
      \fill[gray] \secondcircle;
    \end{scope}

    \begin{scope}
      \clip \secondcircle;
      \fill[gray] \thirdcircle;
    \end{scope}

    \begin{scope}
      \clip \firstcircle;
      \fill[gray] \thirdcircle;
    \end{scope}

    % Next, we want the highlight the intersection of all three circles:

    \begin{scope}
      \clip \firstcircle;
      \clip \secondcircle;
      \fill[black] \thirdcircle;
    \end{scope}



\end{tikzpicture}

% Naturally, all of this could be bundled into nicer macros, but the above 
% should give the idea.

\end{document}

